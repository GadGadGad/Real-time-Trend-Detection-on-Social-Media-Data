\documentclass{beamer}

%========================
% 1) Tiếng Việt (pdfLaTeX)  --- ĐÃ SỬA
%   Dùng gói vietnam theo mẫu bạn đưa, bỏ inputenc + babel
%========================
\usepackage[utf8]{inputenc}
\usepackage[T5]{fontenc}
\usepackage[vietnamese]{babel}

% Font có đủ glyph tiếng Việt (khuyến nghị)
\usepackage{lmodern}
\renewcommand{\familydefault}{\sfdefault}

\renewcommand{\familydefault}{\sfdefault}

%========================
% 2) Theme (giữ Madrid, làm gọn & hiện đại hơn)
%========================
\usetheme{Madrid}

%========================
% 3) Packages cần thiết  --- GIỮ NGUYÊN
%========================
\usepackage{booktabs}
\usepackage{graphicx}
\usepackage{ragged2e}
\usepackage{xcolor}
\usepackage{csquotes}
\usepackage{microtype} % chữ "mượt" hơn (pdfLaTeX)
\usepackage{tikz}
\usetikzlibrary{shapes.geometric, arrows, positioning, fit, calc}

\setbeamertemplate{caption}[numbered]

%========================
% 4) LISTINGS: FIX Unicode tiếng Việt cho *mọi* lstlisting  --- GIỮ NGUYÊN
%========================
\usepackage{listings}

\lstdefinestyle{vncode}{
  basicstyle=\tiny\ttfamily,
  breaklines=true,
  showstringspaces=false,
  columns=fullflexible,
  upquote=true,
  literate=
   {á}{{\'a}}1 {à}{{\`a}}1 {ả}{{\h{a}}}1 {ã}{{\~a}}1 {ạ}{{\d{a}}}1
   {ă}{{\u{a}}}1 {ắ}{{\'{\u{a}}}}1 {ằ}{{\`{\u{a}}}}1 {ẳ}{{\h{\u{a}}}}1 {ẵ}{{\~{\u{a}}}}1 {ặ}{{\d{\u{a}}}}1
   {â}{{\^a}}1 {ấ}{{\'{\^a}}}1 {ầ}{{\`{\^a}}}1 {ẩ}{{\h{\^a}}}1 {ẫ}{{\~{\^a}}}1 {ậ}{{\d{\^a}}}1
   {đ}{{\dj}}1 {Đ}{{\DJ}}1
   {é}{{\'e}}1 {è}{{\`e}}1 {ẻ}{{\h{e}}}1 {ẽ}{{\~e}}1 {ẹ}{{\d{e}}}1
   {ê}{{\^e}}1 {ế}{{\'{\^e}}}1 {ề}{{\`{\^e}}}1 {ể}{{\h{\^e}}}1 {ễ}{{\~{\^e}}}1 {ệ}{{\d{\^e}}}1
   {í}{{\'i}}1 {ì}{{\`i}}1 {ỉ}{{\h{i}}}1 {ĩ}{{\~i}}1 {ị}{{\d{i}}}1
   {ó}{{\'o}}1 {ò}{{\`o}}1 {ỏ}{{\h{o}}}1 {õ}{{\~o}}1 {ọ}{{\d{o}}}1
   {ô}{{\^o}}1 {ố}{{\'{\^o}}}1 {ồ}{{\`{\^o}}}1 {ổ}{{\h{\^o}}}1 {ỗ}{{\~{\^o}}}1 {ộ}{{\d{\^o}}}1
   {ơ}{{\.o}}1 {ớ}{{\'{\.o}}}1 {ờ}{{\`{\.o}}}1 {ở}{{\h{\.o}}}1 {ỡ}{{\~{\.o}}}1 {ợ}{{\d{\.o}}}1
   {ú}{{\'u}}1 {ù}{{\`u}}1 {ủ}{{\h{u}}}1 {ũ}{{\~u}}1 {ụ}{{\d{u}}}1
   {ư}{{\.u}}1 {ứ}{{\'{\.u}}}1 {ừ}{{\`{\.u}}}1 {ử}{{\h{\.u}}}1 {ữ}{{\~{\.u}}}1 {ự}{{\d{\.u}}}1
   {ý}{{\'y}}1 {ỳ}{{\`y}}1 {ỷ}{{\h{y}}}1 {ỹ}{{\~y}}1 {ỵ}{{\d{y}}}1
}

\lstdefinelanguage{json}{
  string=[s]{"}{"},
  stringstyle=\color{blue},
  comment=[l]{:},
  commentstyle=\color{black},
}

\lstset{style=vncode}

%========================
% 5) Biblatex  --- ĐÃ SỬA (mapping)
%   Vì đã bỏ babel[vietnamese], thường KHÔNG cần mapping nữa.
%========================
\usepackage[backend=biber,style=ieee,language=english]{biblatex}
\addbibresource{references.bib}
% \DeclareLanguageMapping{vietnamese}{english} % (không cần nữa khi không dùng babel[vietnamese])
\setbeamertemplate{bibliography item}[text]

%========================
% 6) Table cosmetics  --- GIỮ NGUYÊN
%========================
\renewcommand{\arraystretch}{1.15}


%------------------------------------------------------------
% THÔNG TIN BÀI THUYẾT TRÌNH (GIỮ ĐÚNG TÊN ĐỀ TÀI)
%------------------------------------------------------------
\title[Phát hiện Xu hướng \& Phân loại cảm xúc]
{Phát hiện Xu hướng và Phân loại cảm xúc theo Thời gian thực}

\subtitle{Báo cáo tiến độ dự án - SE363.Q11}

\author[Tăng Nhất, Lê Minh Nhựt]
{\textbf{Thực hiện:} \break Tăng Nhất\inst{1} \and Lê Minh Nhựt\inst{1} \break \textbf{GVHD}: TS. Đỗ Trọng Hợp\inst{2} \break Nguyễn Ngọc Quí\inst{2}}

\institute[VNU-UIT]{
  \inst{1} Khoa Khoa học Máy tính\\
  Trường Đại học Công nghệ Thông tin

  \inst{2} Khoa Khoa học và Kỹ thuật thông tin \\
  Trường Đại học Công nghệ Thông tin
}

\date[\today]{Ngày 16 Tháng 12 năm 2025}

\begin{document}

\frame{\titlepage}

\AtBeginSection[]
{
    \begin{frame}
        \frametitle{Mục lục}
        \tableofcontents[currentsection]
    \end{frame}
}

\begin{frame}
    \frametitle{Mục lục}
    \tableofcontents
\end{frame}

%============================================================
% SECTION 1: GIỚI THIỆU BÀI TOÁN
%============================================================
\section{Giới thiệu bài toán}

% Frame 1.1: Bối cảnh
\begin{frame}{Bối cảnh\& Động lực nghiên cứu}
    \begin{itemize}
        \item \textbf{Sự bùng nổ thông tin:} Mạng xã hội (Facebook) là nơi tin tức lan truyền nhanh nhất, nhưng chứa nhiều nhiễu (tin giả, spam, cảm xúc tiêu cực).
        \item \textbf{Độ trễ của báo chí:} Báo chính thống (News) có độ tin cậy cao nhưng thường chậm hơn sự kiện thực tế (latency).
        \item \textbf{Thách thức ngôn ngữ:} Tiếng Việt có đặc thù phức tạp (tách từ, từ lóng, teencode) gây khó khăn cho các mô hình NLP truyền thống.
        \item \textbf{Nhu cầu thực tế:} Cần một hệ thống \alert{Hybrid (Lai ghép)} kết hợp tốc độ của Social và độ tin cậy của News để phát hiện sự kiện nóng theo thời gian thực.
    \end{itemize}
\end{frame}

% Frame 1.2: Mục tiêu đề tài
\begin{frame}{Mục tiêu đề tài}
    \begin{block}{Mục tiêu chính}
        Xây dựng pipeline tự động phát hiện, phân cụm và tóm tắt xu hướng từ đa nguồn dữ liệu.
    \end{block}

    \vspace{0.5cm}
    \textbf{Các nhiệm vụ cụ thể:}
    \begin{enumerate}
        \item \textbf{Phát hiện sự kiện (Event Detection):} Gom nhóm các bài viết rời rạc thành các chủ đề (Topics) có ý nghĩa.
        \item \textbf{Xử lý nhiễu (Noise Handling):} Lọc bỏ bài viết rác, quảng cáo, nội dung trùng lặp (Near-duplicate).
        \item \textbf{Đặt tên \& Phân loại (Naming \& Taxonomy):} Sử dụng LLM để sinh tiêu đề và phân loại mức độ nghiêm trọng (A/B/C).
        \item \textbf{Phân tích cảm xúc (Sentiment Analysis):} Đánh giá thái độ cộng đồng đối với sự kiện.
    \end{enumerate}
\end{frame}

% Frame 1.3: Placeholder cho Pipeline (Theo yêu cầu của bạn)
\begin{frame}
    \frametitle{Tổng quan quy trình (Pipeline Overview)}
    \begin{center}
    \resizebox{0.75\textwidth}{!}{%
    \begin{tikzpicture}[
        node distance=0.8cm and 0.5cm,
        process/.style={rectangle, draw=blue!60, fill=blue!5, very thick, minimum size=6mm, rounded corners=3mm, font=\small\bfseries, align=center},
        io/.style={trapezium, trapezium left angle=70, trapezium right angle=110, draw=orange!60, fill=orange!10, thick, minimum width=2cm, font=\footnotesize, align=center},
        decision/.style={diamond, draw=red!60, fill=red!10, thick, aspect=2, font=\footnotesize, align=center},
        arrow/.style={thick, ->, >=stealth, color=gray!80},
        label/.style={font=\tiny\itshape, color=gray}
    ]

    % Nodes
    \node (input) [io] {Facebook \& News\\Data Ingestion};
    \node (pre) [process, below=of input] {Preprocessing\\\& Cleaning};
    \node (embed) [process, below=of pre] {Embedding\\(Vietnamese Model)};
    
    \node (sahc) [process, below=of embed, fill=green!10, draw=green!60] {SAHC Clustering\\(Hybrid)};
    
    \node (llm) [process, right=of sahc, xshift=1cm, fill=purple!10, draw=purple!60] {LLM Refinement\\(Gemini)};
    
    \node (vis) [process, above=of llm, yshift=0.5cm] {Dashboard\\\& Visualization};

    % Arrows
    \draw [arrow] (input) -- (pre);
    \draw [arrow] (pre) -- (embed);
    \draw [arrow] (embed) -- (sahc);
    
    \draw [arrow] (sahc) -- node[above, font=\tiny] {Clusters} (llm);
    \draw [arrow] (llm) -- node[right, font=\tiny] {Taxonomy \& Titles} (vis);
    
    % Feedback loop or detail
    \draw [dashed, ->, color=gray] (vis) to[bend right=45] node[left, font=\tiny] {Feedback} (sahc);

    \end{tikzpicture}
    }
    \end{center}
    
    \vspace{0.2cm}
    \begin{itemize} \small
        \item \textbf{1. Ingestion:} Thu thập từ Facebook API \& báo chí.
        \item \textbf{2. Core AI:} Embedding (Vector) + SAHC (Phân cụm lai).
        \item \textbf{3. GenAI:} Gemini sinh tiêu đề và phân loại T1-T7.
    \end{itemize}
\end{frame}

%============================================================
% SECTION 2: DỮ LIỆU
%============================================================
\section{Dữ liệu}

% Frame 2.1: Nguồn dữ liệu
\begin{frame}{Nguồn dữ liệu thu thập}
    Hệ thống thu thập dữ liệu từ 3 nguồn chính để đảm bảo tính đa chiều:

    \begin{columns}
        \column{0.5\textwidth}
        \begin{block}{1. Social Media (Facebook)}
            \begin{itemize}
                \item Các Fanpage/Group cộng đồng lớn.
                \item Đặc điểm: Tốc độ nhanh, dùng tiếng lóng (teencode), nhiều nhiễu, ý kiến cá nhân mạnh.
            \end{itemize}
        \end{block}

        \column{0.5\textwidth}
        \begin{block}{2. Báo chính thống (News)}
            \begin{itemize}
                \item Nguồn: VnExpress, Tuổi Trẻ, Thanh Niên...
                \item Đặc điểm: Chuẩn ngữ pháp, độ tin cậy cao, dùng làm "Anchor" (mỏ neo) xác thực thông tin.
            \end{itemize}
        \end{block}
    \end{columns}

    \vspace{0.3cm}
    \begin{itemize}
        \item \textbf{3. Google Trends:} Sử dụng làm tín hiệu dẫn đường (Signal) để định hướng tìm kiếm.
    \end{itemize}
\end{frame}

% Frame 2.2: Thống kê bộ dữ liệu
\begin{frame}{Thống kê bộ dữ liệu thử nghiệm}
    Dữ liệu được thu thập và gán nhãn sơ bộ để phục vụ quá trình huấn luyện và kiểm thử mô hình:

    \begin{table}[]
        \centering
        \begin{tabular}{l c c}
            \toprule
            \textbf{Nguồn dữ liệu} & \textbf{Số lượng bài đăng} & \textbf{Tỷ lệ (\%)} \\
            \midrule
            Facebook (Social)      & 2,961                      & 38.9\%              \\
            News (Báo chí)         & 4,644                      & 61.1\%              \\
            \midrule
            \textbf{Tổng cộng}     & \textbf{7,605}             & \textbf{100\%}      \\
            \bottomrule
        \end{tabular}
        \caption{Phân bố dữ liệu đầu vào}
    \end{table}

    \vspace{0.2cm}
    \textbf{Quan sát từ dữ liệu:}
    \begin{itemize}
        \item Dữ liệu báo chí chiếm đa số, giúp tạo nền tảng Factual (Sự thật) vững chắc.
        \item Dữ liệu Facebook đủ lớn để phản ánh phản ứng cộng đồng (Sentiment).
    \end{itemize}
\end{frame}

% Frame 2.3: Tiền xử lý dữ liệu (Preprocessing)
\begin{frame}{Thách thức\& Tiền xử lý dữ liệu}
    Dựa trên đặc thù dữ liệu tiếng Việt, quy trình làm sạch bao gồm:

    \begin{enumerate}
        \item \textbf{Lọc nhiễu cơ bản:}
              \begin{itemize}
                  \item Loại bỏ bài viết quá ngắn ($<$ 50 ký tự) hoặc quá dài.
                  \item Loại bỏ các bài viết chứa từ khóa rác (quảng cáo, xổ số, spam).
              \end{itemize}
        \item \textbf{Chuẩn hóa văn bản:}
              \begin{itemize}
                  \item Chuyển đổi Unicode, xử lý emoji, link, hashtag.
                  \item Chuẩn hóa teencode (Facebook data).
              \end{itemize}
        \item \textbf{Tách từ (Word Segmentation):}
              \begin{itemize}
                  \item Sử dụng thư viện chuyên dụng (\texttt{py\_vncorenlp} / \texttt{underthesea}).
                  \item \textbf{Ví dụ:} "đất nước" (1 token) khác với "đất" (soil) và "nước" (water).
                  \item Bước này quan trọng để mô hình Embedding hiểu đúng ngữ cảnh.
              \end{itemize}
    \end{enumerate}
\end{frame}

%============================================================
% SECTION 3: PHƯƠNG PHÁP THỰC HIỆN
%============================================================
\section{Phương pháp thực hiện}

% Frame 3.1: Tổng quan Mô hình đề xuất
\begin{frame}{Kiến trúc Mô hình đề xuất (Proposed Architecture)}
    Hệ thống sử dụng phương pháp tiếp cận \textbf{Hybrid (Lai ghép)}, kết hợp giữa học máy truyền thống và Generative AI:

    \begin{enumerate}
        \item \textbf{Vector hóa (Embedding):} Chuyển đổi văn bản sang không gian vector ngữ nghĩa.
        \item \textbf{Phân cụm lai (SAHC Clustering):} Thuật toán phân cụm phân cấp nhận thức xã hội (Social-Aware Hierarchical Clustering).
        \item \textbf{Tinh chỉnh (Refinement):} Sử dụng LLM để chuẩn hóa kết quả đầu ra.
    \end{enumerate}

    \begin{block}{Công nghệ lõi}
        \begin{itemize}
            \item \textbf{Embedding Model:} \texttt{dangvantuan/vietnamese-document-embedding} (Tối ưu cho tiếng Việt).
            \item \textbf{Clustering:} HDBSCAN (Mật độ) kết hợp cơ chế Anchoring.
            \item \textbf{LLM:} Gemini Pro (Sinh tiêu đề và phân loại).
        \end{itemize}
    \end{block}
\end{frame}

% Frame 3.2: Chi tiết về Embedding
\begin{frame}{Vector hóa văn bản (Text Embedding)}
    \begin{columns}
        \column{0.6\textwidth}
        \textbf{Tại sao chọn model này?}
        \begin{itemize}
            \item Mô hình được huấn luyện trên lượng lớn dữ liệu tiếng Việt.
            \item Khả năng hiểu ngữ cảnh tốt hơn phương pháp từ khóa (TF-IDF).
            \item \textbf{Segmentation:} Áp dụng tách từ trước khi đưa vào mô hình (Ví dụ: "Học sinh" là 1 token thay vì 2).
        \end{itemize}

        \column{0.4\textwidth}
        \begin{center}
            \textbf{[PLACEHOLDER: HÌNH MINH HỌA EMBEDDING SPACE]}
            \par\textit{\small(Các điểm dữ liệu gom lại gần nhau dựa trên ý nghĩa)}
        \end{center}
    \end{columns}

    \vspace{0.5cm}
    \textbf{Quy trình:}
    \texttt{Raw Text} $\to$ \texttt{Clean \& Segment} $\to$ \texttt{Model} $\to$ \texttt{Vector (768 dimensions)}
\end{frame}

% Frame 3.3: Thuật toán SAHC (Rất quan trọng - Logic chính của code)
\begin{frame}{Thuật toán Phân cụm SAHC (Social-Aware Hierarchical Clustering)}
    Quy trình phân cụm diễn ra qua 3 pha để đảm bảo độ chính xác và tính thời sự:

    \begin{itemize}
        \item \textbf{Phase 1: News-First Anchoring (Tạo mỏ neo)}
              \begin{itemize}
                  \item Sử dụng các bài báo chính thống để tạo các tâm cụm (Cluster Centroids) ban đầu.
                  \item Đảm bảo xu hướng có tính xác thực cao.
              \end{itemize}

        \item \textbf{Phase 2: Social Attachment (Gắn kết xã hội)}
              \begin{itemize}
                  \item Tính toán độ tương đồng (Cosine Similarity) của các bài Facebook với các mỏ neo News.
                  \item Nếu \texttt{Score > Threshold}, gán bài Facebook vào cụm News tương ứng.
              \end{itemize}

        \item \textbf{Phase 3: Discovery (Khám phá xu hướng mới)}
              \begin{itemize}
                  \item Các bài Facebook còn lại (chưa được gán) sẽ được chạy thuật toán \textbf{HDBSCAN}.
                  \item Mục tiêu: Phát hiện các sự kiện mới nổi trên MXH mà báo chí chưa kịp đưa tin.
              \end{itemize}
    \end{itemize}
\end{frame}

% Frame 3.4: LLM Refinement Pipeline
\begin{frame}{Hậu xử lý với LLM (LLM Refinement)}
    Sau khi có các cụm thô, hệ thống sử dụng Gemini API để thực hiện 3 tác vụ:

    \begin{enumerate}
        \item \textbf{Summarization \& Naming:} Sinh tiêu đề ngắn gọn, dễ hiểu cho cụm sự kiện (thay vì dùng ID số).
        \item \textbf{Classification (Phân loại Taxonomy):}
              \begin{itemize}
                  \item Gán nhãn sự kiện vào các nhóm A/B/C hoặc T1-T7 (theo mức độ nghiêm trọng/lĩnh vực).
                  \item Loại bỏ các cụm "Rác" (Noise) mà thuật toán clustering bỏ sót.
              \end{itemize}
        \item \textbf{Semantic Deduplication:} Gộp các cụm có cùng ý nghĩa nhưng khác cách diễn đạt (Ví dụ: "Bão Yagi" và "Cơn bão số 3").
    \end{enumerate}
\end{frame}


%============================================================
% SECTION 4: REAL-TIME DATA STREAMING
%============================================================
\section{Real-time Data Streaming}

% Frame 4.1: Kiến trúc luồng dữ liệu (Placeholder)
\begin{frame}{Kiến trúc xử lý thời gian thực (Streaming Architecture)}
    \begin{center}
        \textbf{[PLACEHOLDER: SƠ ĐỒ LUỒNG DỮ LIỆU]}
    \end{center}

    \vspace{0.2cm}
    \textbf{Gợi ý nội dung sơ đồ:}
    \begin{itemize}
        \item \textbf{Data Ingestion:} Các Crawlers chạy song song (đa luồng).
        \item \textbf{Message Queue:} (Ví dụ: Kafka/RabbitMQ hoặc Folder Watching) để đệm dữ liệu.
        \item \textbf{Processing Engine:} Hệ thống phân tích chạy theo chu kỳ (Micro-batch).
        \item \textbf{Storage:} Lưu trữ kết quả vào Database/Cache để hiển thị Dashboard.
    \end{itemize}
\end{frame}

% Frame 4.2: Cơ chế cập nhật (Windowing Strategy)
\begin{frame}{Chiến lược cập nhật dữ liệu (Windowing Strategy)}
    Để xử lý dữ liệu liên tục, hệ thống áp dụng cơ chế cửa sổ thời gian:

    \begin{block}{Cơ chế Sliding Window (Cửa sổ trượt)}
        \begin{itemize}
            \item Hệ thống không chạy Clustering trên từng bài viết đơn lẻ.
            \item Dữ liệu được gom nhóm theo khung thời gian (Ví dụ: mỗi 15 phút hoặc mỗi 100 bài mới).
            \item \textbf{Incremental Update:} Kết hợp dữ liệu mới với các cụm đang "Active" để cập nhật xu hướng mà không cần chạy lại toàn bộ từ đầu.
        \end{itemize}
    \end{block}

    \vspace{0.3cm}
    \textbf{Xử lý trùng lặp (De-duplication):}
    \begin{itemize}
        \item Sử dụng hàm băm (Hashing) nội dung để loại bỏ các bài viết trùng lặp ngay tại đầu vào, giảm tải cho hệ thống.
    \end{itemize}
\end{frame}

\begin{frame}{Tối ưu hóa hiệu năng (Performance Optimization)}
    \textit{(Dựa trên phần Advanced Diagnostics trong code)}
    \begin{itemize}
        \item \textbf{Batch Processing:} Gom nhóm các request gửi lên Embedding Model và LLM để tận dụng khả năng xử lý song song của GPU.
        \item \textbf{Caching:} Lưu trữ các vector embedding đã tính toán (Embeddings Cache) để tránh tính toán lại cho các bài viết cũ.
        \item \textbf{Lọc sớm (Early Filtering):} Loại bỏ rác (Spam/Quảng cáo) bằng từ khóa trước khi đưa vào mô hình AI tốn kém tài nguyên.
    \end{itemize}
\end{frame}

%============================================================
% SECTION 5: HUẤN LUYỆN & TINH CHỈNH (New)
%============================================================
\section{Huấn luyện \& Tinh chỉnh Mô hình}

% Frame 5.1: Xây dựng dữ liệu huấn luyện
\begin{frame}{Chiến lược xây dựng dữ liệu (Data Construction)}
    Hệ thống sử dụng kỹ thuật \textbf{Self-Supervised Learning} và \textbf{Knowledge Distillation} để tạo dữ liệu huấn luyện chất lượng cao mà không tốn nhiều chi phí gán nhãn thủ công:

    \begin{itemize}
        \item \textbf{1. Reranker Data (Contrastive Learning):}
        \begin{itemize}
            \item \textbf{Positive Pairs:} Các bài viết thuộc cùng một cụm (Cluster) được coi là cặp tích cực.
            \item \textbf{Hard Negatives:} Ghép bài viết của cụm A với bài viết của cụm B (khác chủ đề nhưng có thể cùng từ khóa nhiễu) theo tỷ lệ 1:3.
            \item \textit{Tổng mẫu:} ~15,000 cặp huấn luyện.
        \end{itemize}
        
        \item \textbf{2. Taxonomy Data (Distillation):}
        \begin{itemize}
            \item \textbf{Teacher:} Gemini Pro (LLM) gán nhãn T1-T7 cho 1,000 mẫu ngẫu nhiên.
            \item \textbf{Student:} Finetune model \texttt{visobert} trên dữ liệu này để học cách phân loại nhanh (tốc độ gấp 50 lần LLM).
        \end{itemize}
    \end{itemize}
\end{frame}

% Frame 5.2: LLM-as-a-Judge Protocol
\begin{frame}{Phương pháp đánh giá: LLM-as-a-Judge}
    Do thiếu bộ dữ liệu chuẩn (Ground Truth) cho các sự kiện real-time, nhóm áp dụng giao thức đánh giá bằng LLM (Gemini 2.7B) để chấm điểm chất lượng cụm:

    \begin{block}{Giao thức chấm điểm (Scoring Protocol)}
        \begin{enumerate}
            \item \textbf{Input:} Tên cụm + 5 bài viết tiêu biểu (Sample Posts).
            \item \textbf{Criteria (Thang 1-5):}
            \begin{itemize}
                \item \textbf{Coherence (Độ nhất quán):} Các bài viết có cùng nội dung không?
                \item \textbf{Relevance (Độ liên quan):} Tên cụm có phản ánh đúng nội dung không?
                \item \textbf{Distinctiveness (Độ tách biệt):} Cụm này có bị trùng với cụm khác không?
            \end{itemize}
            \item \textbf{Output:} JSON Score + Lý do giải thích (Reasoning).
        \end{enumerate}
    \end{block}

    \vspace{0.2cm}
    \textbf{Ưu điểm:} Đánh giá được ngữ nghĩa sâu (Semantic) mà metrics toán học (Silhouette) không thấy được.
\end{frame}

%============================================================
% SECTION 6: KẾT QUẢ THỰC NGHIỆM
%============================================================
\section{Kết quả thực nghiệm}


% Frame 6.1: So sánh các phiên bản (Experimental Results)
\begin{frame}{So sánh Hiệu năng các mô hình (Model Comparison)}
    Kết quả thực nghiệm trên 3 cấu hình khác nhau (Batch Evaluation):

    \begin{table}[]
        \centering
        \resizebox{0.95\textwidth}{!}{%
        \begin{tabular}{l c c c}
            \toprule
            \textbf{Metrics} & \textbf{Baseline (Hybrid)} & \textbf{Generic Reranker (BGE)} & \textbf{Ours (Finetuned Reranker)} \\
            \midrule
            \textbf{NMI} ($\uparrow$)          & \textbf{0.54} & 0.23 (Poor) & \underline{0.52} (Comparable) \\
            \textbf{BCubed F1} ($\uparrow$)    & \textbf{0.35} & 0.14 & 0.31 \\
            \textbf{Purity} ($\uparrow$)       & \textbf{0.24} & 0.11 & 0.22 \\
            \textbf{Noise Rate} (Filter \%)    & 59\%          & \alert{84\%} (Too Strict) & 57\% (Balanced) \\
            \textbf{Entrooy} ($\downarrow$)    & 4.85          & 6.81 & 5.02 \\
            \bottomrule
        \end{tabular}%
        }
        \caption{So sánh giữa Baseline, Reranker phổ biến (BGE-M3) và Reranker tinh chỉnh}
    \end{table}

    \vspace{0.2cm}
    \textbf{Nhận xét (Insights):}
    \begin{itemize}
        \item \textbf{Generic Reranker (BGE)} hoạt động kém trên dữ liệu tiếng Việt đặc thù (Social Slang), lọc bỏ quá nhiều bài viết hợp lệ (\textbf{84\% Noise}).
        \item \textbf{Finetuned Reranker} phục hồi được hiệu năng gần tương đương Baseline nhưng có khả năng mở rộng tốt hơn cho các tác vụ Ranking sâu hơn.
        \item \textbf{Entropy \& PMI:} Các chỉ số này cho thấy độ gắn kết chủ đề ổn định mặc dù dữ liệu rất nhiễu.
    \end{itemize}
\end{frame}


% Frame 5.2: Đánh giá định lượng (Metrics)
\begin{frame}{Đánh giá định lượng (Quantitative Metrics)}
    Kết quả đánh giá trên tập dữ liệu 7,605 bài đăng:

    \begin{table}[]
        \centering
        \resizebox{0.8\textwidth}{!}{%
        \begin{tabular}{l c c}
            \toprule
            \textbf{Chỉ số (Metric)} & \textbf{Giá trị đạt được} & \textbf{Ý nghĩa}                               \\
            \midrule
            \textbf{NMI}             & \alert{0.54}              & Thông tin tương hỗ so với nhãn gốc (Tốt > 0.5) \\
            \textbf{BCubed F1}       & \alert{0.35}              & Trung bình điều hòa giữa Precision và Recall   \\
            \textbf{Purity}          & \alert{0.25}              & Độ tinh khiết của cụm (tỷ lệ đúng nhãn)        \\
            \textbf{Entropy}         & \alert{4.85}              & Độ hỗn loạn trong cụm (Thấp là tốt)            \\
            \textbf{PMI Coherence}   & -0.61                     & Độ gắn kết ngữ nghĩa giữa các từ trong topic   \\
            \bottomrule
        \end{tabular}%
        }
        \caption{Hiệu năng phân cụm trên tập Mini-Ground Truth (300 mẫu)}
    \end{table}

    \vspace{0.2cm}
    \textbf{So sánh:} Phương pháp Hybrid cải thiện độ chính xác đáng kể so với việc chỉ dùng từ khóa (Keyword Matching) đơn thuần.
\end{frame}

% Frame 5.3: Kết quả tinh chỉnh & Phân loại của LLM
\begin{frame}{Kết quả Phân loại\& Tinh chỉnh (LLM Refinement)}
    Hệ thống tự động sinh tiêu đề và phân cấp sự kiện (Taxonomy):

    \begin{block}{Ví dụ thực tế (Case Study)}
        \begin{itemize}
            \item \textbf{Cụm gốc (Cluster ID 42):} Gồm 150 bài viết chứa từ khóa "gió giật", "mất điện", "cây đổ", "Hà Nội".
            \item \textbf{LLM Refined Title:} \enquote{Siêu bão Yagi gây thiệt hại nghiêm trọng tại Hà Nội}.
            \item \textbf{Phân loại (Taxonomy):} \alert{T1 - Crisis \& Public Risk}.
        \end{itemize}
    \end{block}

    \vspace{0.3cm}
    \textbf{Thống kê các nhóm chủ đề chính (Taxonomy T1-T7):}
    \begin{itemize}
        \item \textbf{T1 - Crisis \& Public Risk:} Thiên tai, sự cố, dịch bệnh (Khẩn cấp).
        \item \textbf{T2 - Policy \& Governance:} Chính sách, pháp luật.
        \item \textbf{T3 - Reputation Risk:} Scandal, khủng hoảng truyền thông.
        \item \textbf{T4/T5 - Market \& Culture:} Xu hướng thị trường, giải trí viral.
    \end{itemize}
\end{frame}

%============================================================
% SECTION 6: DEMO
%============================================================
\section{Demo}

% Frame 6.1: Kịch bản Demo
\begin{frame}{Kịch bản Demo (Demo Scenario)}
    Chúng tôi sẽ trình diễn quy trình xử lý trực tiếp trên \textbf{Interactive Test Bench}:

    \begin{enumerate}
        \item \textbf{Input:} Nạp dữ liệu thô mới nhất từ Facebook Crawler.
        \item \textbf{Processing:}
              \begin{itemize}
                  \item Chạy Pipeline phân cụm (SAHC).
                  \item Kích hoạt LLM để đặt tên cụm.
              \end{itemize}
        \item \textbf{Visualization:} Hiển thị kết quả trên giao diện Dashboard/Notebook.
        \item \textbf{Verification:} Kiểm tra ngẫu nhiên 1 cụm để xem độ chính xác của các bài viết bên trong.
    \end{enumerate}
\end{frame}

% Frame 6.2: Giao diện kết quả (Screenshot)
\begin{frame}{Giao diện Kết quả (Output Visualization)}
    \begin{center}
        % [PLACEHOLDER: Hình ảnh bảng kết quả dạng DataFrame hoặc Dashboard]
        % Có thể dùng ảnh chụp phần "Discovery Viewer" trong code
        \textbf{[PLACEHOLDER: ẢNH CHỤP DISCOVERY VIEWER / DASHBOARD]}
    \end{center}

    \begin{itemize}
        \item Hiển thị danh sách \textbf{Top Trending}.
        \item Phân biệt rõ nguồn tin: \textcolor{blue}{News (Tin cậy)} vs \textcolor{orange}{Social (Lan truyền)}.
        \item Cung cấp tóm tắt ngắn gọn cho người dùng.
    \end{itemize}
\end{frame}

%============================================================
% SECTION 7: KẾT LUẬN VÀ HƯỚNG PHÁT TRIỂN
%============================================================
\section{Kết luận và hướng phát triển}

% Frame 7.1: Kết luận
\begin{frame}{Kết luận}
    \begin{block}{Những kết quả đạt được}
        \begin{enumerate}
            \item Xây dựng thành công pipeline \textbf{Hybrid Event Detection} xử lý đa nguồn dữ liệu (7000+ mẫu thử nghiệm).
            \item Giải quyết tốt bài toán \textbf{nhiễu trên mạng xã hội} thông qua cơ chế Anchoring (gắn kết News).
            \item Tích hợp \textbf{LLM (Gemini)} giúp kết quả đầu ra dễ hiểu, có tính cấu trúc cao thay vì chỉ là danh sách bài viết.
        \end{enumerate}
    \end{block}
\end{frame}

% Frame 7.2: Hạn chế & Hướng phát triển
% Frame 7.2: Thách thức & Bài học kinh nghiệm (Thêm thực tế)
\begin{frame}{Thách thức Kỹ thuật \& Bài học (Challenges)}
    \begin{itemize}
        \item \textbf{1. Semantic Ambiguity (Nhập nhằng ngữ nghĩa):}
              \begin{itemize}
                  \item \textit{Vấn đề:} Cùng một sự kiện nhưng News gọi là "Bão Yagi", Social gọi là "Cơn bão số 3" hoặc "Gió giật Hà Nội".
                  \item \textit{Giải pháp:} Đã áp dụng \textbf{LLM-based Deduplication} kết hợp Reranker để gộp các cụm trùng ý nghĩa.
              \end{itemize}

        \item \textbf{2. Evaluation Gap (Thiếu Ground Truth):}
              \begin{itemize}
                  \item \textit{Vấn đề:} Khó đánh giá chính xác chất lượng phân cụm trên dữ liệu streaming không nhãn.
                  \item \textit{Giải pháp:} Xây dựng quy trình \textbf{"Mini-Ground Truth"} (gán nhãn thủ công 300 mẫu) để chuẩn hoá metrics (BCubed F1, NMI).
              \end{itemize}

        \item \textbf{3. Latency vs. Accuracy Trade-off:}
              \begin{itemize}
                  \item \textit{Vấn đề:} Gọi LLM (Gemini) cho mỗi cụm tốn 2-3s, gây trễ cho pipeline real-time.
                  \item \textit{Giải pháp:} Chiến lược \textbf{Micro-batching} và Caching Embedding; chỉ gọi lại LLM khi cụm thay đổi đáng kể (>20\% posts mới).
              \end{itemize}
    \end{itemize}
\end{frame}

% Frame 7.3: Hướng phát triển
\begin{frame}{Kế hoạch phát triển (Roadmap)}
    \begin{enumerate}
        \item \textbf{Model Quantization:} Chuyển đổi Embedding Model sang ONNX/Int8 để tăng tốc độ inference trên CPU.
        \item \textbf{Advanced Filtering:} Tích hợp bộ lọc "Soft-Noise" để giữ lại các trend giải trí (Meme, Slang) thay vì lọc bỏ hoàn toàn.
        \item \textbf{User Feedback Loop:} Cho phép người dùng gán nhãn lại các cụm sai để Active Learning mô hình phân loại.
        \item \textbf{Scaling:} Triển khai kiến trúc Kafka + Spark Streaming để xử lý >10,000 requests/giây.
    \end{enumerate}
\end{frame}

% Frame cuối: Cảm ơn
\begin{frame}[plain]
    \centering
    \Huge \textcolor{blue}{CẢM ƠN THẦY VÀ CÁC BẠN ĐÃ LẮNG NGHE!}

    \vspace{1cm}
    \large \textbf{Q \& A}
\end{frame}



\end{document}