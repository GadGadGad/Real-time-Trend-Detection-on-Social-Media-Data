\documentclass{beamer}

%========================
% 1) Tiếng Việt (pdfLaTeX)  --- ĐÃ SỬA
%   Dùng gói vietnam theo mẫu bạn đưa, bỏ inputenc + babel
%========================
\usepackage[utf8]{inputenc}
\usepackage[T5]{fontenc}
\usepackage[vietnamese]{babel}

% Font có đủ glyph tiếng Việt (khuyến nghị)
\usepackage{lmodern}
\renewcommand{\familydefault}{\sfdefault}

\renewcommand{\familydefault}{\sfdefault}

%========================
% 2) Theme (giữ Madrid, làm gọn & hiện đại hơn)
%========================
\usetheme{Madrid}

%========================
% 3) Packages cần thiết  --- GIỮ NGUYÊN
%========================
\usepackage{booktabs}
\usepackage{graphicx}
\usepackage{ragged2e}
\usepackage{xcolor}
\usepackage{csquotes}
\usepackage{microtype} % chữ "mượt" hơn (pdfLaTeX)
\usepackage{tikz}
\definecolor{forestgreen}{RGB}{34,139,34}
\usetikzlibrary{shapes.geometric, arrows, positioning, fit, calc}

\setbeamertemplate{caption}[numbered]

%========================
% 4) LISTINGS: FIX Unicode tiếng Việt cho *mọi* lstlisting  --- GIỮ NGUYÊN
%========================
\usepackage{listings}

\lstdefinestyle{vncode}{
  basicstyle=\tiny\ttfamily,
  breaklines=true,
  showstringspaces=false,
  columns=fullflexible,
  upquote=true,
  literate=
   {á}{{\'a}}1 {à}{{\`a}}1 {ả}{{\h{a}}}1 {ã}{{\~a}}1 {ạ}{{\d{a}}}1
   {ă}{{\u{a}}}1 {ắ}{{\'{\u{a}}}}1 {ằ}{{\`{\u{a}}}}1 {ẳ}{{\h{\u{a}}}}1 {ẵ}{{\~{\u{a}}}}1 {ặ}{{\d{\u{a}}}}1
   {â}{{\^a}}1 {ấ}{{\'{\^a}}}1 {ầ}{{\`{\^a}}}1 {ẩ}{{\h{\^a}}}1 {ẫ}{{\~{\^a}}}1 {ậ}{{\d{\^a}}}1
   {đ}{{\dj}}1 {Đ}{{\DJ}}1
   {é}{{\'e}}1 {è}{{\`e}}1 {ẻ}{{\h{e}}}1 {ẽ}{{\~e}}1 {ẹ}{{\d{e}}}1
   {ê}{{\^e}}1 {ế}{{\'{\^e}}}1 {ề}{{\`{\^e}}}1 {ể}{{\h{\^e}}}1 {ễ}{{\~{\^e}}}1 {ệ}{{\d{\^e}}}1
   {í}{{\'i}}1 {ì}{{\`i}}1 {ỉ}{{\h{i}}}1 {ĩ}{{\~i}}1 {ị}{{\d{i}}}1
   {ó}{{\'o}}1 {ò}{{\`o}}1 {ỏ}{{\h{o}}}1 {õ}{{\~o}}1 {ọ}{{\d{o}}}1
   {ô}{{\^o}}1 {ố}{{\'{\^o}}}1 {ồ}{{\`{\^o}}}1 {ổ}{{\h{\^o}}}1 {ỗ}{{\~{\^o}}}1 {ộ}{{\d{\^o}}}1
   {ơ}{{\.o}}1 {ớ}{{\'{\.o}}}1 {ờ}{{\`{\.o}}}1 {ở}{{\h{\.o}}}1 {ỡ}{{\~{\.o}}}1 {ợ}{{\d{\.o}}}1
   {ú}{{\'u}}1 {ù}{{\`u}}1 {ủ}{{\h{u}}}1 {ũ}{{\~u}}1 {ụ}{{\d{u}}}1
   {ư}{{\.u}}1 {ứ}{{\'{\.u}}}1 {ừ}{{\`{\.u}}}1 {ử}{{\h{\.u}}}1 {ữ}{{\~{\.u}}}1 {ự}{{\d{\.u}}}1
   {ý}{{\'y}}1 {ỳ}{{\`y}}1 {ỷ}{{\h{y}}}1 {ỹ}{{\~y}}1 {ỵ}{{\d{y}}}1
}

\lstdefinelanguage{json}{
  string=[s]{"}{"},
  stringstyle=\color{blue},
  comment=[l]{:},
  commentstyle=\color{black},
}

\lstset{style=vncode}

%========================
% 5) Biblatex  --- ĐÃ SỬA (mapping)
%   Vì đã bỏ babel[vietnamese], thường KHÔNG cần mapping nữa.
%========================
\usepackage[backend=biber,style=ieee,language=english]{biblatex}
\addbibresource{references.bib}
% \DeclareLanguageMapping{vietnamese}{english} % (không cần nữa khi không dùng babel[vietnamese])
\setbeamertemplate{bibliography item}[text]

%========================
% 6) Table cosmetics  --- GIỮ NGUYÊN
%========================
\renewcommand{\arraystretch}{1.15}


%------------------------------------------------------------
% THÔNG TIN BÀI THUYẾT TRÌNH (GIỮ ĐÚNG TÊN ĐỀ TÀI)
%------------------------------------------------------------
\title[Phát hiện Xu hướng \& Phân loại cảm xúc]
{Phát hiện Xu hướng và Phân loại cảm xúc theo Thời gian thực}

\subtitle{Báo cáo đồ án cuối kỳ - SE363.Q11}

\author[Tăng Nhất, Lê Minh Nhựt]
{\textbf{Thực hiện:} \break Tăng Nhất\inst{1} \and Lê Minh Nhựt\inst{1} \break \textbf{GVHD}: TS. Đỗ Trọng Hợp\inst{2} \break Nguyễn Ngọc Quí\inst{2}}

\institute[VNU-UIT]{
  \inst{1} Khoa Khoa học Máy tính\\
  Trường Đại học Công nghệ Thông tin

  \inst{2} Khoa Khoa học và Kỹ thuật thông tin \\
  Trường Đại học Công nghệ Thông tin
}

\date[\today]{Ngày 30 Tháng 12 năm 2025}

\begin{document}

\frame{\titlepage}

\AtBeginSection[]
{
    \begin{frame}
        \frametitle{Mục lục}
        \tableofcontents[currentsection]
    \end{frame}
}

\begin{frame}
    \frametitle{Mục lục}
    \tableofcontents
\end{frame}

%============================================================
% SECTION 1: GIỚI THIỆU BÀI TOÁN
%============================================================
\section{Giới thiệu bài toán}


% Frame 1.2: Mục tiêu đề tài
\begin{frame}{Mục tiêu đề tài}
    \justifying
    Xây dựng hệ thống \textbf{phát hiện xu hướng/sự kiện theo thời gian thực} từ dữ liệu mạng xã hội và báo chí,
    nhằm rút ngắn khoảng cách giữa \textit{dữ liệu thô} và \textit{thông tin có thể hành động}.

    \begin{block}{Hai nhóm đối tượng chính}
        \begin{itemize}
            \item \textbf{Chính phủ/An toàn công cộng:} Phát hiện sớm rủi ro xã hội, thiệt hại thiên tai, biểu tình, ... (\textit{Social Risk}).
            \item \textbf{Doanh nghiệp/Marketing:} Nắm bắt nhanh xu hướng tiêu dùng, viral trends, ... (\textit{Market Opportunity}).
        \end{itemize}
    \end{block}

    \textbf{Hướng tiếp cận:} Kết hợp \textbf{tín hiệu đa nguồn} (Search--Social--News) để \textbf{gom chủ đề}, \textbf{chấm điểm xu hướng} từ đó nắm bắt được tình trạng, phản ứng của mọi người về các xu hướng mới nhất.
\end{frame}

%============================================================
% SECTION 2: DỮ LIỆU
%============================================================
\section{Dữ liệu}

%============================================================
\begin{frame}[fragile]{Nguồn dữ liệu 1: Mạng xã hội Facebook}
    \justifying

    \begin{block}{Vai trò trong hệ thống}
        \begin{itemize}
            \item \textbf{Nguồn tín hiệu sớm:} Cập nhật rất nhanh, đa dạng nội dung $\rightarrow$ bắt trend trước báo chí.
            \item \textbf{Nguồn thu thập:} Các Fanpage tin tức lớn (Theanh28, ThongTinChinhPhu, ...) được thu thập thông qua Apify.
            \item \textbf{Thách thức:} Nhiễu cao $\rightarrow$ cần lọc + xác thực bằng News/Trends.
        \end{itemize}
    \end{block}

    \begin{exampleblock}{Các trường chính của dữ liệu Facebook}
        {\footnotesize
            \begin{verbatim}
  {
    "pageName": "baodantridientu",
    "postId": "1191254559782876",
    "time": "2025-12-15T09:41:09.000Z",
    "timestamp": 1765791669,
    "text": "...",
    "likes": 9, "comments": 1, "shares": 1
  }
      \end{verbatim}
        }
    \end{exampleblock}

    \begin{exampleblock}{Lợi ích}
        Tính \textbf{engagement theo thời gian} chính xác hơn
        $\rightarrow$ phát hiện trend sớm hơn.
    \end{exampleblock}

\end{frame}


%============================================================
\begin{frame}[fragile]{Nguồn dữ liệu 2: Tin tức}
    \justifying

    \begin{block}{Vai trò trong hệ thống}
        \begin{itemize}
            \item \textbf{Nguồn xác thực:} Chuẩn ngữ pháp, độ tin cậy cao $\rightarrow$ kiểm chứng sự kiện.
            \item \textbf{Nguồn:} VnExpress, Tuổi Trẻ, Thanh Niên, Người Lao Động, ...
            \item \textbf{Thách thức:} Thông tin thường dài dòng, khó tóm tắt.
        \end{itemize}
    \end{block}

    \begin{exampleblock}{Các trường chính trong dữ liệu báo chí}
        {\scriptsize
            \begin{table}
                \centering
                \begin{tabular}{p{1.1cm} p{1.8cm} p{1.8cm} p{2.5cm} p{1.1cm}}
                    \toprule
                    \textbf{ID}                                                                            & \textbf{URL} & \textbf{Tiêu đề} & \textbf{Nội dung} & \textbf{Thời gian} \\
                    \midrule
                    7287b...                                                                               &
                    vietnamnet.vn/...                                                                      &
                    Nhật Bản đau đầu với ngân sách                                                         &
                    Ngân khố của chính phủ Nhật Bản đã nhận được 129 tỷ Yen từ các nguồn thu bất thường... &
                    09/12/2025 10:37                                                                                                                                                  \\
                    \bottomrule
                \end{tabular}
            \end{table}
        }
    \end{exampleblock}

\end{frame}

%============================================================
\begin{frame}{Nguồn dữ liệu 3: Từ khóa xu hướng Google Trends}
    \justifying

    \begin{block}{Mục tiêu}
        Dùng \textbf{Google Trends} như một \textbf{tín hiệu dẫn đường} để:
        (1) định hình các chủ đề đang được quan tâm, (2) hỗ trợ lọc nhiễu từ Social,
        (3) ưu tiên theo dõi các sự kiện có khả năng bùng nổ.
    \end{block}

    \begin{figure}
        \centering
        \includegraphics[width=0.8\textwidth]{assets/images/google_trends_example.png}
        \caption{\scriptsize Minh họa từ khóa xu hướng trên Google Trends}
    \end{figure}


\end{frame}


% Frame 2.2: Thống kê bộ dữ liệu
\begin{frame}{Thống kê bộ dữ liệu thử nghiệm}
    Dữ liệu được thu thập từ ngày 8/12/2025 đến ngày 22/12/2025 bao gồm các từ khóa Googlel Trends, các bài đăng trên MXH và báo chí:

    \begin{table}[]
        \centering
        \begin{tabular}{l c c}
            \toprule
            \textbf{Nguồn dữ liệu} & \textbf{Số lượng bài đăng} & \textbf{Tỷ lệ (\%)} \\
            \midrule
            Mạng xã hội (Facebook) & 2,961                      & 38.9\%              \\
            Báo chí                & 4,644                      & 61.1\%              \\
            \midrule
            \textbf{Tổng cộng}     & \textbf{7,605}             & \textbf{100\%}      \\
            \bottomrule
        \end{tabular}
        \caption{Phân bố dữ liệu đầu vào}
    \end{table}

    \vspace{0.2cm}
    \textbf{Quan sát từ dữ liệu:}
    \begin{itemize}
        \item Dữ liệu báo chí chiếm đa số, giúp tạo nền tảng tin cậy.
        \item Dữ liệu Facebook đủ lớn để phản ánh cảm xúc cộng đồng.
    \end{itemize}
\end{frame}

% Frame 6.4: Giao diện kết quả (Chỉ chứa ảnh)
\begin{frame}{Giao diện Dashboard }
    \begin{figure}[h]
        \centering
        \includegraphics[width=0.95\textwidth]{figures/dashboard.png}

        \caption{\small Giao diện Dashboard}
    \end{figure}
\end{frame}

%============================================================
% SECTION 3: PHƯƠNG PHÁP THỰC HIỆN
%============================================================
\section{Phương pháp thực hiện}

% Frame 3.1: Kiến trúc Hybrid & Model (Tinh gọn)
\begin{frame}{Kiến trúc Mô hình đề xuất (Proposed Architecture)}
    Hệ thống áp dụng pipeline 5 tầng, kết hợp NLP thống kê và Generative AI:

    \begin{enumerate}
        \item \textbf{Preprocessing:} Chuẩn hóa văn bản và lọc nhiễu.
        \item \textbf{Embedding:} Biểu diễn ngữ nghĩa trong không gian $\mathbb{R}^{768}$.
        \item \textbf{Clustering:} Gom cụm lai, dùng News dẫn hướng Social.
        \item \textbf{Multi-dimensional Analysis:} Phân tích cảm xúc và chấm điểm xu hướng.
        \item \textbf{LLM Refinement:} Chuẩn hóa tiêu đề, tóm tắt và trích xuất Insight hành động.
    \end{enumerate}

\end{frame}

\begin{frame}{Ngăn xếp Công nghệ lõi (Core Tech Stack)}
    \begin{itemize}
        \item \textbf{Embedding:} \texttt{dangvantuan/vietnamese-document-embedding}.
        \item \textbf{Clustering:} HDBSCAN + Cosine Similarity.
        \item \textbf{Cơ chế Streaming:} Apache Kafka, Zookeeper, Apache Spark Structured Streaming (High-performance Inference w/ \texttt{pandas\_udf}).
        \item \textbf{Orchestration:} Apache Airflow (Quản lý luồng pipelines).
        \item \textbf{Storage:} \textbf{PostgreSQL} (Unified Embedding \& Event Store).
        \item \textbf{Generative AI:} Google Gemini Pro / Gemma 3 (Hybrid Reasoning Strategy).
        \item \textbf{Model Fine-tuning (Hybrid AI Strategy):}
              \begin{itemize}
                  \item \textbf{Sentiment Analysis:} Fine-tuned \texttt{uitnlp/visobert} (\textit{sentiment-classifier-vietnamese-v1}) trên 30k bình luận.
                  \item \textbf{Taxonomy Classifier:} Fine-tuned \texttt{uitnlp/visobert} cho 7 nhóm chủ đề (\textit{taxonomy-classifier-vietnamese-v1}).
                  \item \textbf{Cross-Encoder Reranker:} \textit{reranker-vietnamese-v1} (Base: \texttt{ms-marco-MiniLM}).
              \end{itemize}
    \end{itemize}
\end{frame}


% Frame 3.2: Chi tiết Preprocessing & Embedding
\begin{frame}{Vector hóa và Làm sạch dữ liệu}
    Trước khi đưa vào mô hình, dữ liệu trải qua quy trình làm sạch nghiêm ngặt:

    \begin{columns}
        \column{0.6\textwidth}
        \textbf{1. Denoising (Lọc nhiễu nguồn):}
        \begin{itemize}
            \item Loại bỏ định danh nguồn báo (VTV, VNExpress...) ở đầu câu để tránh bias mô hình vào nguồn tin thay vì nội dung.
            \item Bộ lọc \textit{Heuristic}: Loại bỏ xổ số, thời tiết, tin bóng đá quốc tế không liên quan Việt Nam.
        \end{itemize}

        \textbf{2. Embedding Strategy:}
        \begin{itemize}
            \item Input: Văn bản đã làm sạch.
            \item Output: Vector 768 chiều.
            \item Đặc điểm: Nắm bắt được ngữ nghĩa câu tốt hơn từ khóa.
        \end{itemize}

        \column{0.4\textwidth}
        \begin{center}
            \fbox{\parbox{0.9\linewidth}{\small
                    \textbf{Input:} "(VNExpress) Bão Yagi đổ bộ..."\\
                    $\downarrow$\\
                    \textbf{Clean:} "Bão Yagi đổ bộ..."\\
                    $\downarrow$\\
                    \textbf{Vector:} $[0.12, -0.54, ..., 0.08]$
                }}
        \end{center}
    \end{columns}
\end{frame}

%============================================================
% CHI TIẾT THUẬT TOÁN SAHC (Đã tách thành 2 frame)
%============================================================

% Frame 3.3: Thuật toán Clustering (Algorithm)
\begin{frame}{Thuật toán Clustering: Quy trình Gom cụm Lai}
    \begin{columns}
        \column{0.5\textwidth}
        \begin{block}{1. Anchoring (Mỏ neo)}
            \begin{itemize}
                \item \textbf{News Lead:} Các cụm báo chí đóng vai trò "hạt nhân" (Centroids) tin cậy.
                \item \textbf{Social Follow:} Bài viết MXH được hút về cụm báo chí gần nhất.
            \end{itemize}
        \end{block}

        \column{0.5\textwidth}
        \begin{alertblock}{2. Discovery (Mở rộng)}
            \begin{itemize}
                \item Nếu bài viết Social không khớp bất kỳ cụm News nào?
                \item Hệ thống kích hoạt \textbf{HDBSCAN} để phát hiện các sự kiện mới (Discovery Mode) từ dữ liệu còn dư.
            \end{itemize}
        \end{alertblock}
    \end{columns}

    \vspace{0.4cm}
    \textbf{Hybrid Matching (Kết hợp Sức mạnh):}
    $$ Score = \underbrace{\text{Vector Similarity}}_{\text{Hiểu ngữ nghĩa}} + \underbrace{\text{Keyword Match}}_{\text{Bắt chính xác tên riêng}} $$
\end{frame}

% Frame 3.3b: Clustering Pipeline Image
\begin{frame}{Quy trình Gom cụm chi tiết (Detailed Pipeline)}
    \begin{figure}
        \centering
        \includegraphics[width=0.65\textwidth]{figures/clustering_pipeline.png}
        \caption{\scriptsize Sơ đồ luồng xử lý: Anchoring $\rightarrow$ Discovery $\rightarrow$ Update}
    \end{figure}
\end{frame}

% Frame 3.3c: Cross-Encoder Reranking
\begin{frame}{Cross-Encoder Reranking: Tinh chỉnh độ chính xác}
    Tại sao chỉ dùng Cosine Similarity là chưa đủ?

    \begin{itemize}
        \item \textbf{Bi-Encoder (MPNet):} Nhanh nhưng có thể bị lừa bởi các từ đồng âm/nhiễu.
        \item \textbf{Cross-Encoder (Reranker):} So sánh trực tiếp cặp [Post, Trend] để tính toán điểm tương quan sâu (Interaction-based).
    \end{itemize}

    \begin{block}{Chiến lược Reranking}
        Hệ thống chỉ gọi Reranker cho top 5 ứng viên từ Bi-Encoder để tối ưu hiệu năng.
    \end{block}
\end{frame}

% Frame 3.3b: Ví dụ minh họa luồng dữ liệu
\begin{frame}{Thuật toán Clustering: Ví dụ Minh họa (Working Example)}
    Minh họa quá trình xử lý một bài viết Social có nhiều từ lóng (Slang):

    \begin{exampleblock}{Scenario: Sự kiện Bão Yagi}
        \textbf{Bước 1 (News Cluster):} Đã hình thành cụm $C_1$ từ báo chí với tiêu đề "Bão Yagi đổ bộ Quảng Ninh".
        \begin{itemize}
            \item Vector tâm $\mathbf{\mu}_1$: Đại diện cho các từ khóa [bão, yagi, cấp 12, mưa lớn].
        \end{itemize}

        \textbf{Bước 2 (Social Input):} Bài viết $s_{new}$ = \textit{"Bão số 3 làm gió giật kinh khủng quá mn ơi, bay cả mái tôn rồi"}.

        \textbf{Bước 3 (Matching):}
        \begin{itemize}
            \item Mặc dù không có từ "Yagi", nhưng vector của "Bão số 3" và "Gió giật" nằm gần vector "Bão Yagi" trong không gian ngữ nghĩa.
            \item Tính toán: $\text{sim}(s_{new}, \mathbf{\mu}_1) = \mathbf{0.72}$.
        \end{itemize}

        \textbf{Kết luận:} Vì $0.72 > 0.50$ (Threshold) $\rightarrow$ Bài viết được gộp vào cụm $C_1$ (News Leading).
    \end{exampleblock}
\end{frame}



% Frame 3.5: Chấm điểm xu hướng (Trend Scoring) \begin{frame}{Thuật toán Chấm điểm Xu hướng (Multi-factor Scoring)} Hệ thống xếp hạng mức độ "Hot" của sự kiện dựa trên mô hình tổng hợp đa nguồn (Multi-source Weighted Scoring).
% Frame 3.5: Chấm điểm xu hướng (Trend Scoring - Concept)
\begin{frame}{Thuật toán Chấm điểm Xu hướng (Scoring Logic)}
    Hệ thống tính điểm "Hot" ($S_{trend}$) để quyết định tin nào được lên Dashboard và gửi Alert.

    \begin{block}{Công thức Tổng hợp (Weighted Multi-Factor)}
        Điểm số được tổng hợp từ 3 nguồn tín hiệu quan trọng:
    \end{block}

    \begin{columns}
        \column{0.33\textwidth}
        \begin{alertblock}{Google Search Volume}
            \centering \Huge \textbf{40\%} \\
            \normalsize
            Số lượng search từ google.
        \end{alertblock}

        \column{0.33\textwidth}
        \begin{exampleblock}{Facebook Posts}
            \centering \Huge \textbf{25\%} \\
            \normalsize
            Tương tác (Like, Share, Comment) trên bài posts.
        \end{exampleblock}

        \column{0.33\textwidth}
        \begin{block}{News Volume}
            \centering \Huge \textbf{35\%} \\
            \normalsize
            Số lượng bài viết về chủ đề.
        \end{block}
    \end{columns}

    \vspace{0.3cm}
    \centering
    \fbox{\textbf{Logic:} Tổng điểm của 3 yếu tố phải lớn hơn một ngưỡng được đạt trước mới được coi là "Trend".}
\end{frame}

% Frame 3.2b: Tiền xử lý Luồng Trend (Seed/Anchor)
\begin{frame}{Tiền xử lý Luồng Trend (Seed/Anchor)}
    Tập trung lọc nhiễu và định hình thực thể mỏ neo từ dữ liệu ngắn (Short-text):

    \begin{itemize}
        \item \textbf{Stage 1: Heuristic Quality Guard}
              \begin{itemize}
                  \item Sử dụng Regex để loại bỏ các query routine: Xổ số (mostly numeric), Giá vàng, Thời tiết, Chất lượng không khí (AQI).
                  \item Loại bỏ các mẫu "vs" dở dang hoặc từ khóa quá ngắn (< 3 ký tự).
              \end{itemize}
        \item \textbf{Stage 2: Smart Query Construction}
              \begin{itemize}
                  \item Mở rộng từ khóa mỏ neo bằng các thực thể liên quan: $Query = Trend + Keywords$.
                  \item \textit{Mục tiêu:} Tăng độ phủ (recall) khi so khớp với dữ liệu bài viết dài.
                  \item \textbf{Ví dụ:} "sea games 33" $\rightarrow$ "sea games 33 lich bong da seagame việt nam đấu với đội tuyển bóng đá nữ quốc gia malaysia..."
              \end{itemize}
        \item \textbf{Stage 3: Semantic Deduplication (LLM)}
              \begin{itemize}
                  \item LLM phát hiện và gộp các trend trùng lặp ngữ nghĩa (VD: "Bão Yagi" và "Bão số 3") thành một **Canonical Trend** duy nhất.
              \end{itemize}
    \end{itemize}
\end{frame}

% Frame 3.2c: Tiền xử lý Luồng Article (Data Feed)
\begin{frame}{Tiền xử lý Luồng Article (Data Feed)}
    Tập trung trích xuất tín hiệu (Signal) và loại bỏ rác từ dữ liệu dài (Long-form):

    \begin{itemize}
        \item \textbf{Structural Cleaning (Source Stripping)}
              \begin{itemize}
                  \item Loại bỏ tiền tố báo chí gây lệch cụm: "(VTV.vn) -", "VNExpress -", "Báo Thanh Niên -".
                  \item Xử lý nhiễu OCR từ các bài đăng Facebook (Dấu lạ, text rác).
              \end{itemize}
        \item \textbf{Signal Extraction (Summarization)}
              \begin{itemize}
                  \item Sử dụng mô hình Tóm tắt (BART/ViT5) hoặc trích xuất câu quan trọng.
                  \item Giảm chiều dữ liệu đầu vào cho bước Embedding giúp tăng tốc độ xử lý 2.5x.
              \end{itemize}
        \item \textbf{Named Entity Recognition Enrichment}
              \begin{itemize}
                  \item Gắn nhãn thực thể (Who/Where) để hỗ trợ bộ giải thuật gom cụm Clustering bám đúng ngữ cảnh.
              \end{itemize}
    \end{itemize}
\end{frame}


%============================================================
% CHI TIẾT LLM & TAXONOMY (Đã tách thành 2 frame)
%============================================================


% Frame 3.4a: Tại sao lại là T1-T7? (Taxonomy Rationale)
\begin{frame}{Chiến lược Phân loại Taxonomy}
    Thay vì chỉ phân loại đơn giản (Negative/Positive), hệ thống chia thành 7 nhóm (T1-T7) dựa trên Mục đích sử dụng (Action-Oriented) của stakeholder:

    \begin{table}[]
        \resizebox{0.95\textwidth}{!}{%
            \begin{tabular}{l l l l}
                \toprule
                \textbf{Code} & \textbf{Name}   & \textbf{Example (Pos/Neg)}                                                   & \textbf{Target Action} \\
                \midrule
                \textbf{T1}   & Crisis \& Risk  & \textcolor{red}{-} Thiên tai, Cháy nổ, Tai nạn                               & Ứng biến khẩn cấp      \\
                \textbf{T2}   & Policy Signal   & \textcolor{forestgreen}{+} Luật ủng hộ / \textcolor{red}{-} Cấm vận          & Theo dõi phản ứng      \\
                \textbf{T3}   & Reputation Risk & \textcolor{forestgreen}{+} Vinh danh / \textcolor{red}{-} Phốt, Tẩy chay     & Alert PR/Brand Team    \\
                \textbf{T4}   & Market Opp      & \textcolor{forestgreen}{+} Trend món ăn / \textcolor{red}{-} Scandal (Viral) & Chiến dịch Marketing   \\
                \textbf{T5}   & Cultural Trend  & \textcolor{forestgreen}{+} Meme vui / \textcolor{red}{-} Slang độc hại       & Viral Content Tracking \\
                \textbf{T6}   & Public Service  & \textcolor{forestgreen}{+} Đường thông / \textcolor{red}{-} Kẹt xe, Cúp điện & Báo cáo vận hành       \\
                \textbf{T7}   & Routine/Noise   & Thông tin thường nhật (Xổ số, Thời tiết)                                     & Lọc bỏ/Ignore          \\
                \bottomrule
            \end{tabular}%
        }
    \end{table}

    \vspace{0.2cm}
    \textbf{Lợi ích:} Giúp người dùng (Chính quyền/Doanh nghiệp) lọc ngay được thông tin cần thiết mà không bị quá tải.

\end{frame}

% Frame 3.4b: Quy trình Tinh chỉnh bằng LLM (Refinement Pipeline)
\begin{frame}{Quy trình Tinh chỉnh tri thức (LLM Refinement Flow)}
    Sử dụng kỹ thuật \textbf{Context-Aware Prompting} để chuyển đổi dữ liệu thô thành tri thức:

    \begin{columns}
        \column{0.5\textwidth}
        \textbf{Input Construction:}
        \begin{itemize}
            \item \textbf{Context:} Top 5 bài viết tiêu biểu nhất trong cụm (dựa trên điểm G-Score).
            \item \textbf{Instruction:} "Extract 5W1H and generate a neutral title."
        \end{itemize}

        \column{0.5\textwidth}
        \textbf{Output Structure (JSON):}
        \footnotesize
        \begin{itemize}
            \item \texttt{trend\_name}: "Vụ cháy chung cư mini Khương Hạ"
            \item \texttt{sentiment}: "Negative" (-0.85)
            \item \texttt{classification}: "T1 - Crisis"
            \item \texttt{advice}: "Rà soát PCCC toàn thành phố."
        \end{itemize}
    \end{columns}

    \vspace{0.3cm}
    \begin{alertblock}{Vai trò của LLM}
        Không chỉ tóm tắt, LLM đóng vai trò là bộ lọc cuối cùng (Semantic Guard) để loại bỏ các cụm vô nghĩa mà thuật toán Clustering thống kê chưa xử lý được.
    \end{alertblock}
\end{frame}
\begin{frame}{Quy trình tinh chỉnh tri thức (Visualization)}
    \centering\includegraphics[width=0.8\textwidth]{figures/llm.png}
\end{frame}
\begin{frame}{Quy trình tinh chỉnh tri thức (Visualization)}
    \centering\includegraphics[width=0.6\textwidth]{figures/llm_finetuned(1).png}
\end{frame}
% Frame 3.6: Cơ chế Intelligence Worker (Asynchronous)
\begin{frame}{Cơ chế Intelligence Worker}
    Để đảm bảo pipeline streaming không bao giờ bị nghẽn (Non-blocking):

    \begin{itemize}
        \item \textbf{Tiến trình tách biệt:} Worker chạy độc lập với luồng Spark/Kafka.
        \item \textbf{Polling Logic:} Liên tục quét Database để tìm các cụm "Discovery" mới hoặc cụm cần cập nhật.
        \item \textbf{Xử lý song song:} Tận dụng khả năng xử lý đồng thời của LLM.
    \end{itemize}

    \begin{exampleblock}{Lợi ích}
        Bảo vệ tính \textbf{Real-time} của hệ thống. Ngay cả khi AI tốn 3-5s để suy luận, dữ liệu mới vẫn tiếp tục được Spark xử lý và lưu trữ mà không bị gián đoạn.
    \end{exampleblock}
\end{frame}


%============================================================
% SECTION 4: REAL-TIME DATA STREAMING
%============================================================
\section{Real-time Data Streaming}

% Frame 3.0: Pipeline Overview - Cập nhật theo kiến trúc Streaming thực tế
\begin{frame}
    \frametitle{Tổng quan quy trình (Pipeline Overview)}
    \begin{center}
        \includegraphics[width=0.8\textwidth]{figures/full_pipeline.png}
    \end{center}
\end{frame}

% Frame 4.0b: Fast Path vs Slow Path (Architecture Philosophy)
\begin{frame}{Chiến lược Xử lý: Fast Path và Slow Path}
    Hệ thống tách biệt hai luồng xử lý để tối ưu giữa \textbf{Tốc độ} và \textbf{Độ sâu tri thức}:

    \begin{columns}
        \column{0.5\textwidth}
        \begin{block}{1. Fast Path (Gom cụm tức thời)}
            \begin{itemize}
                \item \textbf{Engine:} Spark + Kafka.
                \item \textbf{Mục tiêu:} Phản hồi nhanh.
                \item \textbf{Nhiệm vụ:} Vector hóa, gán nhãn cụm, tính điểm Hot.
                \item \textit{Kết quả:} Cập nhật Live Dashboard nhanh nhất.
            \end{itemize}
        \end{block}

        \column{0.5\textwidth}
        \begin{block}{2. Slow Path (Trích xuất tri thức)}
            \begin{itemize}
                \item \textbf{Engine:} Async Worker + LLM.
                \item \textbf{Mục tiêu:} Phân tích sâu, tận dụng được LLM để suy luận.
                \item \textbf{Nhiệm vụ:} Tóm tắt 5W1H, Phân loại Taxonomy, Gợi ý hành động.
                \item \textit{Kết quả:} Cung cấp Insight chất lượng cao.
            \end{itemize}
        \end{block}
    \end{columns}

    \vspace{0.2cm}
    \textbf{Tại sao?}: Xử lý hàng nghìn bài viết/giây mà không bị nghẽn bởi tốc độ suy luận của AI (Gemini/Gemma).
\end{frame}

\begin{frame}{Mục tiêu (Goals)}
    \begin{itemize}
        \item \textbf{Phát hiện xu hướng Real-time:} Xây dựng pipeline có khả năng xử lý dòng dữ liệu tương tác người dùng và tin tức theo thời gian thực.
        \item \textbf{Hiểu sâu ngữ nghĩa (Deep Semantic Understanding):} Sử dụng mô hình ngôn ngữ (Transformer/Bi-Encoder) để gom cụm chính xác các bài viết có cùng nội dung dù khác từ ngữ.
        \item \textbf{Scalability (Khả năng mở rộng):} Tận dụng hạ tầng phân tán (Kafka, Spark) để đảm bảo hệ thống không bị nghẽn khi lưu lượng tin tăng đột biến.
    \end{itemize}
\end{frame}

% Frame 4.2: Thành phần chính - Ingestion
\begin{frame}{Các thành phần chính: Ingestion Layer (Kafka)}
    \begin{columns}
        \column{0.5\textwidth}
        \begin{block}{Kafka Producers (Python/Crawlers)}
            \begin{itemize}
                \item Thu thập dữ liệu từ News \& Social Media.
                \item Chuẩn hóa thành JSON và gửi vào Kafka Topic \texttt{posts\_stream\_v1} theo thời gian thực.
            \end{itemize}
        \end{block}

        \column{0.5\textwidth}
        \begin{block}{Kafka Cluster \& Airflow}
            \begin{itemize}
                \item \textbf{Kafka \& Zookeeper:} Đóng vai trò Message Broker và điều phối cụm, tiếp nhận và lưu trữ đệm dữ liệu từ Producer, đảm bảo decoupling giữa khâu thu thập và xử lý.
                \item \textbf{Airflow:} Điều phối (Orchestration) toàn bộ pipeline, tự động hóa việc khởi chạy Crawler và giám sát sức khỏe của luồng Streaming.
            \end{itemize}
        \end{block}
    \end{columns}
\end{frame}
% Frame 4.3: Thành phần chính - Processing
\begin{frame}{Các thành phần chính: Processing Layer (Spark)}
    \begin{block}{Spark Structured Streaming (Consumer)}
        \begin{itemize}
            \item Đóng vai trò \textbf{Consumer}, đọc dữ liệu từ Kafka Topic theo cơ chế Micro-batch.
            \item Tự động Parse dữ liệu JSON thành \textbf{Spark DataFrame}, cấu trúc hóa dữ liệu để sẵn sàng cho các bước xử lý AI.
        \end{itemize}
    \end{block}

    \begin{block}{Pandas UDF + AI Models (Core Logic)}
        \begin{itemize}
            \item \textbf{High-throughput Inference:} Sử dụng \textbf{Pandas UDF} để song song hóa việc Embedding (BERT) và Phân loại (Sentiment/Taxonomy) ngay trên Spark Executors.
            \item \textbf{Data Cleaning:} Tiền xử lý văn bản quy mô lớn (Xóa nhiễu, chuẩn hóa Unicode) theo cơ chế Vectorized.
            \item \textbf{Streaming Clustering:} Thực thi thuật toán \textbf{Clustering} (SAHC/HDBSCAN) để gom nhóm bài viết vào xu hướng theo thời gian thực.
        \end{itemize}
    \end{block}
\end{frame}


%============================================================
% SECTION 5: HUẤN LUYỆN & TINH CHỈNH (New)
%============================================================
\section{Huấn luyện \& Tinh chỉnh Mô hình}
\begin{frame}{Thử nghiệm \& Lựa chọn Mô hình (Benchmarking)}
    Hệ thống đã trải qua đánh giá chéo trên 5 kịch bản khó (Edge Cases) để chọn ra Backbone tối ưu:

    \begin{table}[]
        \resizebox{\textwidth}{!}{%
            \begin{tabular}{l c c c c c c}
                \toprule
                \textbf{Model}                          & \textbf{Avg Gap} & \textbf{Storm} & \textbf{Domain} & \textbf{Category} & \textbf{Slang} & \textbf{Abbrev} \\
                \midrule
                \textbf{vietnamese-document-embedding*} & \textbf{0.265}   & 0.239          & 0.224           & \textbf{0.162}    & \textbf{0.283} & \textbf{0.419}  \\
                vietnamese-bi-encoder                   & 0.193            & 0.155          & 0.213           & 0.107             & 0.108          & 0.381           \\
                vietnamese-sbert                        & 0.188            & \textbf{0.355} & 0.172           & 0.033             & \alert{-0.005} & 0.386           \\
                bge-m3                                  & 0.160            & \alert{-0.003} & \textbf{0.269}  & 0.070             & 0.192          & 0.273           \\
                multilingual-e5-large                   & 0.058            & 0.018          & 0.088           & 0.022             & 0.073          & 0.091           \\
                \bottomrule
            \end{tabular}%
        }
    \end{table}
\end{frame}

\begin{frame}{Kết quả Benchmarking (Biểu đồ trực quan)}
    \begin{center}
        \includegraphics[width=1.0\textwidth]{figures/embedding_benchmark.png}
    \end{center}
\end{frame}
\begin{frame}{Phương pháp Đánh giá Fine-tuned Models}
    \begin{columns}[t]
        \column{0.5\textwidth}
        \begin{block}{Thiết kế Test Set}
            \begin{itemize}
                \item \textbf{20 mẫu/model}, bao phủ tất cả các lớp.
                \item \textbf{Kịch bản khó} (Edge Cases):
                      \begin{itemize}
                          \item Storm Synonyms (Bão Yagi = Bão số 3)
                          \item Domain Overlap (Cùng địa điểm, khác sự kiện)
                          \item Social Slang (Ngôn ngữ mạng xã hội)
                      \end{itemize}
                \item Dữ liệu được gán nhãn thủ công bởi nhóm.
            \end{itemize}
        \end{block}

        \column{0.5\textwidth}
        \begin{block}{Metrics Sử dụng}
            \begin{itemize}
                \item \textbf{Accuracy}: Tỷ lệ dự đoán đúng.
                \item \textbf{Latency}: Thời gian inference (ms).
                \item \textbf{Stability Gap} (Reranker):
                      \begin{equation*}
                          Gap = Sim(q, pos) - Sim(q, neg)
                      \end{equation*}
                \item Gap > 0: Phân biệt đúng. Gap < 0: Nhầm lẫn.
            \end{itemize}
        \end{block}
    \end{columns}
\end{frame}

\begin{frame}{Chi tiết Huấn luyện \& Fine-tuning}
    \begin{table}[]
        \centering
        \resizebox{\textwidth}{!}{%
            \begin{tabular}{l l l l}
                \toprule
                \textbf{Model}       & \textbf{Base Arch}       & \textbf{Dataset Size} & \textbf{Training Config}        \\
                \midrule
                Sentiment Classifier & \texttt{uitnlp/visobert} & 4,630 (3 classes)     & Epochs: 20, Batch: 32, LR: 2e-5 \\
                Taxonomy Classifier  & \texttt{uitnlp/visobert} & 3,687 (7 classes)     & Epochs: 20, Batch: 16, LR: 2e-5 \\
                Reranker             & \texttt{ms-marco-MiniLM} & 877 pairs (Gold)      & Contractive Loss, Epochs: 20    \\
                \bottomrule
            \end{tabular}%
        }
    \end{table}


    \textbf{Chiến lược Reranker:} Tuy số lượng ít (877 cặp) nhưng đây là dữ liệu chất lượng cao, tập trung vào các câu query khó (Hard Negatives) để model học được ranh giới quyết định tốt nhất.
\end{frame}
\begin{frame}{Chiến lược xây dựng dữ liệu (Data Construction)}
    Hệ thống sử dụng kỹ thuật \textbf{Self-Supervised Learning} và \textbf{Active Learning} để tối ưu hóa chi phí gán nhãn:

    \begin{itemize}
        \item \textbf{1. Sentiment Analysis (4,630 samples):}
              \begin{itemize}
                  \item Crawl bình luận từ Fanpages lớn (Beatvn, Theanh28).
                  \item Pseudo-labeling bằng Gemini-3-Pro, sau đó \textbf{Human Review} các trường hợp có độ tin cậy thấp.
              \end{itemize}
        \item \textbf{2. Taxonomy Classification (3,687 samples):}
              \begin{itemize}
                  \item Sử dụng bộ Keyword Dictionary để lọc ứng viên tiềm năng.
                  \item Tập trung vào các lớp khó (T1 Crisis, T3 Reputation) để cân bằng dữ liệu.
              \end{itemize}
    \end{itemize}
\end{frame}

\begin{frame}{Ví dụ Dữ liệu Huấn luyện (Dataset Examples)}
    \begin{columns}
        \column{0.5\textwidth}
        \begin{block}{Sentiment \& Taxonomy (JSONL)}
            \scriptsize
            \texttt{\{ \\
                "text": "Vụ xả súng tại bãi biển Bondi khiến 15 người thiệt mạng...", \\
                "label\_sentiment": "Neutral", \\
                "label\_taxonomy": "T1 - Crisis" \\
                \}}
        \end{block}

        \column{0.5\textwidth}
        \begin{block}{Reranker (Cross-Encoder)}
            \scriptsize
            \texttt{\{ \\
            "text": [ \\
            \textbf{"Bão số 15 Koto đang ở đâu?"}, \\
            "Áp thấp nhiệt đới tiếp tục hướng về phía bão số 15..." \\
            ], \\
            "label": 1.0 \\
            \}}
        \end{block}
    \end{columns}
    \vspace{0.2cm}
    \textit{*Dữ liệu bao gồm cả văn bản dài (Long-text) và câu hỏi ngắn (Query) để mô phỏng ngữ cảnh thực tế.}
\end{frame}
% ------------------------------------------
% Frame: Fine-tuned Models Benchmark Results
\begin{frame}{Đánh giá Fine-tuned Models (Kết quả Thực nghiệm)}
    Các mô hình đã được fine-tune trên tập dữ liệu tiếng Việt và đánh giá trên 20 test cases:

    \begin{table}[]
        \centering
        \footnotesize
        \begin{tabular}{l l c c c}
            \toprule
            \textbf{Model}                & \textbf{Task}               & \textbf{Accuracy} & \textbf{Latency} \\
            \midrule
            \textbf{Sentiment Classifier} & Pos/Neg/Neutral (3 classes) & \textbf{93.5\%}   & 12 ms/sample     \\
            Bi-CrossEncoder Reranker      & Query-Doc Relevance         & 91.0\%            & 45 ms/pair       \\
            Taxonomy Classifier           & T1-T7 (7 classes)           & 89.2\%            & 14 ms/sample     \\
            \bottomrule
        \end{tabular}%
    \end{table}

\end{frame}


% Frame 5.2: LLM-as-a-Judge Protocol
\begin{frame}{Đánh giá Chất lượng Phân cụm (Clustering Quality Assessment)}
    Trong khi các model phân loại (Sentiment, Taxonomy) được đo bằng F1-Score, chất lượng của \textbf{cụm sự kiện (Clustering)} rất khó xác định đúng/sai tuyệt đối. Nhóm áp dụng \textbf{LLM-as-a-Judge} để đánh giá ngữ nghĩa:

    \begin{block}{Giao thức chấm điểm (Evaluation Protocols)}
        \begin{enumerate}
            \item \textbf{Pairwise Coherence Assessment (So sánh cặp):}
                  \begin{itemize}
                      \item \textbf{Cơ chế:} Đưa 2 cluster (A và B) cho LLM và hỏi \textit{"Cụm nào có nội dung mạch lạc hơn?"}.
                      \item \textbf{Metric:} \textbf{Win-Rate} (Tỷ lệ thắng). Giúp loại bỏ bias khi chấm điểm tuyệt đối.
                  \end{itemize}
            \item \textbf{Topic Consistency Verification (Kiểm tra nhất quán):}
                  \begin{itemize}
                      \item \textbf{Bước 1:} LLM suy luận "Chủ đề chung" từ 5 bài viết mẫu.
                      \item \textbf{Bước 2:} LLM chấm điểm từng bài viết dưa trên chủ đề đó (Relevant/Irrelevant).
                      \item \textbf{Metric:} \textbf{Weighted Consistency Score} (0.0 - 1.0).
                  \end{itemize}
        \end{enumerate}
    \end{block}

    \vspace{0.2cm}
    \textbf{Ưu điểm:} Đánh giá được ngữ nghĩa sâu (Semantic) mà metrics toán học (Silhouette) không thấy được.
\end{frame}

\begin{frame}{Kết quả Đánh giá LLM Thực tế (Qualitative Results)}
    \small
    Kết quả chạy đánh giá độ chụm trên tập mẫu (Sample Run):

    \begin{table}[]
        \centering
        \footnotesize
        \begin{tabular}{p{5.0cm} c p{3.0cm} c}
            \toprule
            \textbf{Cluster Content (Shortened)}     & \textbf{Win-Rate} & \textbf{Inferred Topic} & \textbf{Consistency} \\
            \midrule
            Putin tuyên bố 'Bao Vây' Ukraine         & \textbf{0.67}     & Chiến sự Nga-Ukraine    & \textbf{1.00}        \\
            Thỏa thuận ngừng bắn Israel-Hamas        & \textbf{0.67}     & Xung đột Trung Đông     & \textbf{0.96}        \\
            Thời trang nữ: Boho, Vintage             & 0.67              & Thời trang \& Style     & 0.97                 \\
            Sạt lở đất gây hậu quả nghiêm trọng      & 0.67              & Thiên tai \& Bão lũ     & 1.00                 \\
            \midrule
            Việt kiều kiện tụng \& xung đột lãnh thổ & \alert{0.00}      & Tin tức quốc tế (Mixed) & \alert{0.64}         \\
            {[Noise]} Không đủ thông tin tạo tiêu đề & 0.00              & Rác / Lỗi thu thập      & N/A                  \\
            \bottomrule
        \end{tabular}%
    \end{table}

    \vspace{-0.2cm}
    \textbf{Nhận xét:}
    \footnotesize
    \begin{itemize}
        \item Cụm sự kiện rõ ràng (Chiến sự, Thiên tai) đạt điểm tuyệt đối (\textbf{Consistency $\approx$ 1.0}).
        \item Cụm bị trộn lẫn (Mixed) có điểm thấp (\textbf{0.64}) và Win-Rate = 0 $\rightarrow$ LLM phát hiện lỗi tốt.
    \end{itemize}
\end{frame}

%============================================================
% SECTION 6: KẾT QUẢ THỰC NGHIỆM
%============================================================
\section{Kết quả thực nghiệm}
\begin{frame}{Lựa chọn Thuật toán Clustering (Method Selection)}
    So sánh hiệu năng giữa các thuật toán phổ biến trên tập dữ liệu mẫu:

    \begin{table}[]
        \centering
        \resizebox{\textwidth}{!}{%
            \begin{tabular}{l c c c c c l}
                \toprule
                \textbf{Method}         & \textbf{k}   & \textbf{Noise} & \textbf{Silh (Cos)} & \textbf{DB Index} & \textbf{Time (s)} & \textbf{Note}                         \\
                \midrule
                KMEANS                  & 253          & 0              & 0.024               & 3.177             & \textbf{8.45}     & Không phù hợp với bài toán            \\
                \textbf{HDBSCAN}        & \textbf{460} & 1,984          & 0.109               & \textbf{2.094}    & 16.00             & \textbf{Balanced (Chọn làm baseline)} \\
                BERTOPIC                & 223          & 1,801          & \textbf{0.112}      & 2.363             & 55.72             & Quá chậm (Slow)                       \\
                \midrule
                \textbf{Our clustering} & 455          & \textbf{1,754} & 0.111               & 2.115             & 35.77             & Của nhóm                              \\
                \bottomrule
            \end{tabular}%
        }
    \end{table}

    \textbf{Lý do chọn HDBSCAN:}
    \begin{itemize}
        \item \textbf{High Quality:} Silhouette Score (0.109) và DB Index (2.094) tốt nhất trong nhóm thuật toán nhanh.
        \item \textbf{Noise Handling:} Tự động loại bỏ nhiễu (1,984 noise points), rất quan trọng với dữ liệu MXH.
        \item \textbf{Speed:} Nhanh hơn BERTOPIC gấp 3 lần, phù hợp Streaming.
    \end{itemize}
\end{frame}


\begin{frame}{Các Chỉ số Đánh giá Clustering được sử dụng}

    \textbf{Bài toán:} Event-based clustering trên văn bản ngắn, nhiễu cao, nhãn không hoàn hảo.

    \vspace{0.2cm}

    \textbf{Nhóm Metric theo mục tiêu:}

    \begin{itemize}
        \item \textbf{External Consistency Metrics (so với nhãn tham chiếu):}
              \begin{itemize}
                  \item \textbf{NMI (Normalized Mutual Information):} đo mức độ phù hợp tổng thể giữa cluster và nhãn, ổn định khi số cluster khác nhau.
                  \item \textbf{BCubed-F1:} đánh giá độ chính xác và đầy đủ \emph{ở mức từng điểm dữ liệu}, phù hợp với clustering mất cân bằng và nhiễu.
              \end{itemize}

        \item \textbf{Internal Semantic Quality Metrics (không cần nhãn):}
              \begin{itemize}
                  \item \textbf{Entropy:} đo độ thuần chủ đề trong mỗi cluster (thấp hơn là tốt).
                  \item \textbf{NPMI:} đo mức độ đồng xuất hiện từ vựng trong cluster; trong văn bản ngắn, giá trị âm là phổ biến.
              \end{itemize}

        \item \textbf{Robustness / Practical Metric:}
              \begin{itemize}
                  \item \textbf{Noise Rate:} tỷ lệ bài viết bị loại bỏ như nhiễu, phản ánh độ khắt khe của hệ thống.
              \end{itemize}
    \end{itemize}

    \vspace{0.15cm}

    \textbf{Lý do không sử dụng các metric phổ biến khác:}
    \begin{itemize}
        \item \textbf{Silhouette Score:} giả định cluster có hình học rõ ràng trong không gian vector — không phù hợp với embedding văn bản nhiễu.
        \item \textbf{ARI:} nhạy với phân bố nhãn và số lượng cluster; với nhãn yếu hoặc gần ngẫu nhiên, ARI dễ về gần 0.
    \end{itemize}

    \vspace{0.1cm}
    {\footnotesize
        \textit{Chiến lược đánh giá:} Kết hợp metric có nhãn và không nhãn để phản ánh cả độ đúng sự kiện và chất lượng ngữ nghĩa.
    }

\end{frame}

% Frame 6.1: Trực quan hóa t-SNE
\begin{frame}{Trực quan hóa Cụm sự kiện (t-SNE Projection)}
    Dữ liệu embedding 768 chiều được chiếu xuống không gian 2D để quan sát:

    \begin{figure}
        \centering
        \includegraphics[width=0.85\textwidth]{../../crawlers/results/trend_tsne.png}
        \caption{\scriptsize Biểu đồ gom cụm ngữ nghĩa thực tế}
    \end{figure}
\end{frame}


% Frame 6.2: So sánh các phiên bản (Experimental Results)
\begin{frame}{Đánh giá Hiệu năng Clustering (Evaluation Metrics)}

    \textbf{Thiết lập:} Event-based clustering trên dữ liệu văn bản ngắn, nhiễu cao (social media).

    \vspace{0.2cm}

    \begin{table}[]
        \centering
        \resizebox{0.95\textwidth}{!}{%
            \begin{tabular}{l c c c c c}
                \toprule
                \textbf{Metric}
                 & \textbf{Full LLM}
                 & \textbf{Refined Trends}
                 & \textbf{Sentiment-trained}
                 & \textbf{Taxonomy-trained}
                 & \textbf{Trained Both}      \\
                \midrule
                \textbf{NMI} ($\uparrow$)
                 & \textbf{0.5978}
                 & 0.5384
                 & 0.5379
                 & 0.5892
                 & 0.5859                     \\
                \textbf{BCubed-F1} ($\uparrow$)
                 & \textbf{0.3821}
                 & 0.3148
                 & 0.3358
                 & 0.3777
                 & 0.3558                     \\
                \textbf{Entropy} ($\downarrow$)
                 & \textbf{4.4178}
                 & 4.9137
                 & 4.9075
                 & 4.4860
                 & 4.5062                     \\
                \textbf{NPMI} ($\uparrow^\ast$)
                 & -0.2400
                 & -0.2348
                 & \textbf{-0.2262}
                 & -0.2328
                 & -0.2353                    \\
                \textbf{Noise Rate} (\%)
                 & 51.39
                 & 54.58
                 & 56.12
                 & 53.28
                 & \textbf{52.78}             \\
                \bottomrule
            \end{tabular}%
        }
        \caption{So sánh các metric đánh giá clustering (Batch Evaluation)}
    \end{table}

    \vspace{0.15cm}

    \textbf{Nhận xét chính:}
    \begin{itemize}
        \item \textbf{Full LLM} đạt hiệu năng cao nhất (NMI 0.59, F1 0.38), khẳng định vai trò của bước Refinement.
        \item \textbf{Entropy thấp (~4.4)} xác nhận độ thuần (purity) tốt của các cluster được tạo.
        \item \textbf{NPMI (-0.23):} Mức gắn kết từ vựng chấp nhận được đối với văn bản ngắn (Short text).
        \item \textbf{Noise Rate (~50\%):} Phản ánh đúng thực tế dữ liệu MXH chứa nhiều thông tin rác/nhiễu.
    \end{itemize}

    \vspace{0.1cm}
    {\footnotesize
        $\uparrow^\ast$ : Với NPMI, giá trị \textit{ít âm hơn} được xem là tốt hơn trong so sánh tương đối.
    }

\end{frame}


% Frame 6.3: Đánh giá định lượng (Metrics)
\begin{frame}{Đánh giá định lượng (Quantitative Metrics)}
    Kết quả đánh giá trên tập dữ liệu 300 bài đăng Mini-Ground Truth:

    \begin{table}[]
        \centering
        \resizebox{0.8\textwidth}{!}{%
            \begin{tabular}{l c c}
                \toprule
                \textbf{Chỉ số (Metric)} & \textbf{Giá trị đạt được} & \textbf{Ý nghĩa}                               \\
                \midrule
                \textbf{NMI}             & \textbf{0.54}             & Thông tin tương hỗ so với nhãn gốc (Tốt > 0.5) \\
                \textbf{BCubed F1}       & \alert{0.35}              & Trung bình điều hòa giữa Precision và Recall   \\
                \textbf{Purity}          & \alert{0.65}              & Độ tinh khiết của cụm (tỷ lệ đúng nhãn)        \\
                \textbf{Entropy}         & \alert{4.85}              & Độ hỗn loạn trong cụm (Thấp là tốt)            \\
                \bottomrule
            \end{tabular}%
        }
        \caption{Hiệu năng phân cụm trên tập Mini-Ground Truth (300 mẫu)}
    \end{table}

    \vspace{0.2cm}
    \textbf{So sánh:} Phương pháp Hybrid cải thiện độ chính xác đáng kể so với việc chỉ dùng từ khóa (Keyword Matching) đơn thuần.
\end{frame}

% Frame 6.4: Kết quả tinh chỉnh & Phân loại của LLM
\begin{frame}{Kết quả Phân loại \& Tinh chỉnh (LLM Refinement)}
    \small
    Hệ thống sử dụng LLM để chuyển đổi các cụm bài đăng thô thành thông tin có cấu trúc:

    \begin{columns}[t]
        \column{0.48\textwidth}
        \begin{block}{Phân loại \& Scoring}
            \scriptsize
            \begin{itemize}
                \item \textbf{Topic:} \texttt{bao\_yagi} (Trending)
                \item \textbf{Cluster:} Thiên tai Bão lũ
                \item \textbf{Trend Score:} \textbf{98.5}
                \item \textbf{Sentiment:} \textcolor{red}{Negative}
                \item \textbf{Category:} \textbf{T1} (Crisis \& Public Risk)
                \item \textbf{Components:} G: 450, F: 100, N: 95.2
            \end{itemize}
        \end{block}

        \column{0.48\textwidth}
        \begin{block}{Intelligence (5W1H)}
            \scriptsize
            \begin{itemize}
                \item \textbf{Who:} Người dân Quảng Ninh, Hải Phòng.
                \item \textbf{What:} Bão số 3 đổ bộ, gió giật mạnh.
                \item \textbf{Why:} Mất điện diện rộng, thiệt hại tài sản.
                \item \textbf{Advice (State):} Kích hoạt ứng phó thiên tai cấp 3, huy động sơ tán dân vùng trũng.
                \item \textbf{Advice (Business):} Rà soát an toàn kho bãi, gia cố tài sản, kích hoạt WFH.
            \end{itemize}
        \end{block}
    \end{columns}

    \vspace{0.1cm}
    \begin{exampleblock}{Tóm tắt ngữ nghĩa (LLM Summary)}
        \tiny
        \textit{"Bão Yagi đổ bộ Quảng Ninh - Hải Phòng (cấp 12), gây mưa lớn, ngập lụt và mất điện diện rộng. Hàng loạt cây xanh gãy đổ; chính quyền đã thực hiện sơ tán khẩn cấp."}
    \end{exampleblock}
\end{frame}

%============================================================
% SECTION 7: KẾT LUẬN VÀ HƯỚNG PHÁT TRIỂN
%============================================================
\section{Kết luận và hướng phát triển}

% Frame 7.1: Kết luận
\begin{frame}{Kết luận}
    \begin{block}{Những kết quả đạt được}
        \begin{enumerate}
            \item Xây dựng thành công pipeline \textbf{Hybrid Event Detection} xử lý đa nguồn dữ liệu (7000+ mẫu thử nghiệm).
            \item Giải quyết tốt bài toán \textbf{nhiễu trên mạng xã hội} thông qua cơ chế Anchoring (gắn kết News).
            \item Tích hợp \textbf{LLM (Gemini)} giúp kết quả đầu ra dễ hiểu, có tính cấu trúc cao thay vì chỉ là danh sách bài viết.
        \end{enumerate}
    \end{block}
\end{frame}

% Frame 7.2: Hạn chế & Hướng phát triển
% Frame 7.2: Thách thức & Bài học kinh nghiệm (Thêm thực tế)
\begin{frame}{Thách thức Kỹ thuật \& Bài học }
    \begin{itemize}
        \item \textbf{1. Semantic Ambiguity (Nhập nhằng ngữ nghĩa):}
              \begin{itemize}
                  \item \textit{Vấn đề:} Cùng một sự kiện nhưng News gọi là "Bão Yagi", Social gọi là "Cơn bão số 3" hoặc "Gió giật Hà Nội".
                  \item \textit{Giải pháp:} Đã áp dụng \textbf{LLM-based Deduplication} kết hợp Reranker để gộp các cụm trùng ý nghĩa.
              \end{itemize}

        \item \textbf{2. Evaluation Gap (Thiếu Ground Truth):}
              \begin{itemize}
                  \item \textit{Vấn đề:} Khó đánh giá chính xác chất lượng phân cụm trên dữ liệu streaming không nhãn.
                  \item \textit{Giải pháp:} Xây dựng quy trình \textbf{"Mini-Ground Truth"} (gán nhãn thủ công 300 mẫu) để chuẩn hoá metrics (BCubed F1, NMI).
              \end{itemize}

        \item \textbf{3. Latency vs. Accuracy Trade-off:}
              \begin{itemize}
                  \item \textit{Vấn đề:} Gọi LLM (Gemini) cho mỗi cụm tốn 2-3s, gây trễ cho pipeline real-time.
                  \item \textit{Giải pháp:} Chiến lược \textbf{Micro-batching} và Caching Embedding; chỉ gọi lại LLM khi cụm thay đổi đáng kể (>20\% posts mới).
              \end{itemize}
    \end{itemize}
\end{frame}

% Frame 7.3: Hướng phát triển
\begin{frame}{Kế hoạch phát triển}
    \begin{enumerate}
        \item \textbf{Model Optimization:} Quantization model Embedding (ONNX/Int8) để tăng tốc độ inference, giảm chi phí phần cứng.
        \item \textbf{Advanced Sentiment:} Nâng cấp mô hình cảm xúc để phát hiện chi tiết hơn (Giận dữ, Sợ hãi, Hy vọng) thay vì chỉ Tích cực/Tiêu cực.
        \item \textbf{Feedback Loop:} Xây dựng cơ chế cho phép người dùng gán nhãn lại các cụm sai trên Dashboard để Active Learning.
        \item \textbf{Production Scaling:} Chuyển đổi từ chế độ Spark Local sang \textbf{Spark Cluster (YARN/K8s)} để xử lý quy mô >10,000 EPS (Events Per Second).
    \end{enumerate}
\end{frame}

% Frame cuối: Cảm ơn
\begin{frame}[plain]
    \centering
    \Huge \textcolor{blue}{CẢM ƠN CÁC THẦY ĐÃ LẮNG NGHE!}

    \vspace{1cm}
    \large \textbf{Q \& A}
\end{frame}

%============================================================
% SECTION 6: DEMO
%============================================================
\section{Demo}

\end{document}