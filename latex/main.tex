\documentclass{beamer}

%========================
% 1) Tiếng Việt (pdfLaTeX)
%========================
\usepackage[utf8]{inputenc}
\usepackage[T5]{fontenc}
\usepackage[vietnamese]{babel}

% Font sans cho slide
\usepackage{helvet}
\renewcommand{\familydefault}{\sfdefault}

%========================
% 2) Theme
%========================
\usetheme{Madrid}
\usecolortheme{default}

%========================
% 3) Packages cần thiết
%========================
\usepackage{booktabs}
\usepackage{graphicx}
\usepackage{ragged2e}
\usepackage{xcolor}
\usepackage{csquotes}

% (caption trong beamer đôi khi gây phiền; nếu không thật sự cần, nên bỏ)
% \usepackage{caption}
\setbeamertemplate{caption}[numbered]

%========================
% 4) LISTINGS: FIX Unicode tiếng Việt cho *mọi* lstlisting
%========================
\usepackage{listings}

% Map tiếng Việt dùng chung (áp cho mọi listings, không chỉ JSON)
\lstdefinestyle{vncode}{
  basicstyle=\tiny\ttfamily,
  breaklines=true,
  showstringspaces=false,
  columns=fullflexible,
  upquote=true,
  literate=
   {á}{{\'a}}1 {à}{{\`a}}1 {ả}{{\h{a}}}1 {ã}{{\~a}}1 {ạ}{{\d{a}}}1
   {ă}{{\u{a}}}1 {ắ}{{\'{\u{a}}}}1 {ằ}{{\`{\u{a}}}}1 {ẳ}{{\h{\u{a}}}}1 {ẵ}{{\~{\u{a}}}}1 {ặ}{{\d{\u{a}}}}1
   {â}{{\^a}}1 {ấ}{{\'{\^a}}}1 {ầ}{{\`{\^a}}}1 {ẩ}{{\h{\^a}}}1 {ẫ}{{\~{\^a}}}1 {ậ}{{\d{\^a}}}1
   {đ}{{\dj}}1 {Đ}{{\DJ}}1
   {é}{{\'e}}1 {è}{{\`e}}1 {ẻ}{{\h{e}}}1 {ẽ}{{\~e}}1 {ẹ}{{\d{e}}}1
   {ê}{{\^e}}1 {ế}{{\'{\^e}}}1 {ề}{{\`{\^e}}}1 {ể}{{\h{\^e}}}1 {ễ}{{\~{\^e}}}1 {ệ}{{\d{\^e}}}1
   {í}{{\'i}}1 {ì}{{\`i}}1 {ỉ}{{\h{i}}}1 {ĩ}{{\~i}}1 {ị}{{\d{i}}}1
   {ó}{{\'o}}1 {ò}{{\`o}}1 {ỏ}{{\h{o}}}1 {õ}{{\~o}}1 {ọ}{{\d{o}}}1
   {ô}{{\^o}}1 {ố}{{\'{\^o}}}1 {ồ}{{\`{\^o}}}1 {ổ}{{\h{\^o}}}1 {ỗ}{{\~{\^o}}}1 {ộ}{{\d{\^o}}}1
   {ơ}{{\.o}}1 {ớ}{{\'{\.o}}}1 {ờ}{{\`{\.o}}}1 {ở}{{\h{\.o}}}1 {ỡ}{{\~{\.o}}}1 {ợ}{{\d{\.o}}}1
   {ú}{{\'u}}1 {ù}{{\`u}}1 {ủ}{{\h{u}}}1 {ũ}{{\~u}}1 {ụ}{{\d{u}}}1
   {ư}{{\.u}}1 {ứ}{{\'{\.u}}}1 {ừ}{{\`{\.u}}}1 {ử}{{\h{\.u}}}1 {ữ}{{\~{\.u}}}1 {ự}{{\d{\.u}}}1
   {ý}{{\'y}}1 {ỳ}{{\`y}}1 {ỷ}{{\h{y}}}1 {ỹ}{{\~y}}1 {ỵ}{{\d{y}}}1
}

% JSON language (chỉ thêm màu sắc, còn Unicode đã được style vncode lo)
\lstdefinelanguage{json}{
  string=[s]{"}{"},
  stringstyle=\color{blue},
  comment=[l]{:},
  commentstyle=\color{black},
}

% Áp style mặc định cho tất cả lstlisting
\lstset{style=vncode}

%========================
% 5) Biblatex (fix cảnh báo vietnamese.lbx)
%========================
\usepackage[backend=biber,style=ieee,language=english]{biblatex}
\addbibresource{references.bib}
\DeclareLanguageMapping{vietnamese}{english}
\setbeamertemplate{bibliography item}[text]

%------------------------------------------------------------
% THÔNG TIN BÀI THUYẾT TRÌNH
%------------------------------------------------------------
\title[Phát hiện Xu hướng \& Phân tích Tâm lý]
{Phát hiện Xu hướng và Phân tích Tâm lý theo Thời gian thực}

\subtitle{Báo cáo tiến độ dự án - SE363.Q11}

\author[Tăng Nhất, Lê Minh Nhựt]
{\textbf{Thực hiện:} \break Tăng Nhất\inst{1} \and Lê Minh Nhựt\inst{1} \break \textbf{GVHD}: TS. Đỗ Trọng Hợp\inst{1}}

\institute[VNU-UIT]{
  \inst{1} Khoa Khoa học Máy tính\\
  Trường Đại học Công nghệ Thông tin
}

\date[\today]{Tháng 12 năm 2025}

%------------------------------------------------------------
\begin{document}

\frame{\titlepage}

\AtBeginSection[]
{
  \begin{frame}
    \frametitle{Mục lục}
    \tableofcontents[currentsection]
  \end{frame}
}

\begin{frame}
  \frametitle{Nội dung báo cáo}
  \tableofcontents
\end{frame}

%------------------------------------------------------------
\section{Tổng quan dự án (Overview)}
\begin{frame}{Mục tiêu đề tài}
  \justifying
  Xây dựng hệ thống \textbf{Real-time Event Detection} trên mạng xã hội nhằm thu hẹp khoảng cách giữa dữ liệu thô và thông tin chi tiết có thể hành động.

  \begin{block}{Hai nhóm đối tượng chính}
    \begin{itemize}
      \item \textbf{Chính phủ/An toàn công cộng:} Phát hiện sớm các rủi ro xã hội, biểu tình, tin giả (Social Risk).
      \item \textbf{Doanh nghiệp/Marketing:} Nắm bắt nhanh các xu hướng tiêu dùng, viral trends (Market Opportunity).
    \end{itemize}
  \end{block}

  \textbf{Phương pháp tiếp cận:}
  Kết hợp Học bán giám sát (Semi-Supervised Learning) để xử lý dữ liệu stream thiếu nhãn.
\end{frame}

\begin{frame}{Phân loại sự kiện (Taxonomy)}
  Hệ thống tập trung vào 7 loại sự kiện chính:
  \begin{enumerate}
    \item \textbf{Social Controversy:} Tranh cãi xã hội, bê bối (Ưu tiên cao cho CP).
    \item \textbf{Civil Unrest:} Biểu tình, đình công.
    \item \textbf{Natural Disaster:} Thiên tai, lũ lụt, môi trường.
    \item \textbf{Public Safety:} Tai nạn, cháy nổ.
    \item \textbf{Politics \& Policy:} Bầu cử, phản ứng chính sách.
    \item \textbf{Viral Lifestyle:} Xu hướng ăn uống, thời trang (Ưu tiên cho Marketing).
    \item \textbf{Entertainment:} Giải trí, văn hóa đại chúng.
  \end{enumerate}
\end{frame}

%------------------------------------------------------------
\section{Dữ liệu \& Thu thập (Data Collection)}
\begin{frame}[fragile]{Nguồn dữ liệu 1: Mạng xã hội (Social Media)}
  \begin{itemize}
    \item \textbf{Nguồn:} Các Fanpage lớn (Ví dụ: Theanh28, Beatvn...)
    \item \textbf{Định dạng:} JSON (Content, Media, Stats).
    \item \textbf{Trạng thái:} Đã hoàn thành crawl.
  \end{itemize}

  \begin{exampleblock}{Mẫu dữ liệu JSON (Crawl từ Facebook)}
\begin{lstlisting}[language=json]
{
  "page_name": "Theanh28",
  "published_time": "2025-12-15T01:54:10",
  "content": "Vào 100 đám cưới xin đồ ăn nuôi 120 mèo hoang...",
  "stats": { "likes": 3800, "comments": 99 }
}
\end{lstlisting}
  \end{exampleblock}
\end{frame}

\begin{frame}[fragile]{Nguồn dữ liệu 2: Tin tức (News)}
  \begin{itemize}
    \item \textbf{Nguồn:} Báo Thanh Niên, Tuổi Trẻ, VnExpress, Vietnamnet...
    \item \textbf{Định dạng:} CSV (Title, Description, Content, Author).
    \item \textbf{Mục đích:} Dùng làm dữ liệu nền (baseline) hoặc kiểm chứng thông tin.
  \end{itemize}

  \begin{exampleblock}{Mẫu dữ liệu CSV}
\begin{lstlisting}
article_id, url, title, content
"7287b...", "vietnamnet.vn/...", "Nhật Bản đau đầu...",
"Ngân khố của chính phủ Nhật Bản đã nhận được 129 tỷ Yen..."
\end{lstlisting}
  \end{exampleblock}
\end{frame}

%------------------------------------------------------------
\section{Phương pháp \& Kiến trúc (Methodology)}
\begin{frame}{Kiến trúc hệ thống dự kiến}
  Hệ thống chia làm 2 pha chính:
  \begin{columns}
    \column{0.5\textwidth}
    \textbf{Phase 1: Offline Discovery}
    \begin{itemize}
      \item Crawl dữ liệu quá khứ.
      \item Tiền xử lý \& Vector hóa.
      \item Phân cụm (Clustering).
      \item Gán nhãn bán giám sát (Label Propagation).
    \end{itemize}

    \column{0.5\textwidth}
    \textbf{Phase 2: Online Real-time}
    \begin{itemize}
      \item Streaming Ingestion (Kafka).
      \item Inference thời gian thực.
      \item Cảnh báo \& Lưu trữ.
    \end{itemize}
  \end{columns}
\end{frame}

\begin{frame}{Thay đổi \& Cập nhật mô hình}
  Sau quá trình thử nghiệm (08/12/2025), nhóm đã thực hiện các điều chỉnh:

  \begin{alertblock}{Vấn đề gặp phải}
    Dữ liệu "Daily basis" quá nhiều (Dự báo thời tiết hàng ngày, giá vàng, xổ số...) gây nhiễu lớn.
  \end{alertblock}

  \begin{block}{Giải pháp điều chỉnh}
    \begin{itemize}
      \item \textbf{Thay đổi Model:} Chuyển từ mini-LLM sang \textbf{PhoBERT} để xử lý tiếng Việt tốt hơn.
      \item \textbf{Bổ sung NER:} Trích xuất thực thể để lọc nội dung rác và gom nhóm chính xác hơn.
    \end{itemize}
  \end{block}
\end{frame}

%------------------------------------------------------------
\section{Khó khăn \& Kế hoạch (Challenges \& Plan)}
\begin{frame}{Khó khăn hiện tại}
  \begin{enumerate}
    \item \textbf{Vấn đề Phân cụm (Clustering):}
      \begin{itemize}
        \item Chất lượng phân cụm chưa cao.
        \item Cụm bị chồng lấn bởi bài viết tương tự (bán hàng, spam) nhưng không cùng sự kiện.
      \end{itemize}
    \item \textbf{Xử lý nhiễu (Noise filtering):}
      \begin{itemize}
        \item Chưa có cơ chế triệt để loại bỏ tin lặp hàng ngày (giá vàng, thời tiết) khỏi luồng sự kiện nóng.
      \end{itemize}
  \end{enumerate}
\end{frame}

\begin{frame}{Kế hoạch tiếp theo (Timeline)}
  \begin{table}
    \centering
    \begin{tabular}{l l l}
      \toprule
      \textbf{Task} & \textbf{Mô tả} & \textbf{Trạng thái} \\
      \midrule
      Task 1 & Phân tích yêu cầu \& Thiết kế & Đang làm \\
      Task 2 & Thu thập dữ liệu (Crawler) & \textbf{Hoàn thành} \\
      Task 3 & Nghiên cứu NLP (PhoBERT/NER) & TBD \\
      Task 4 & Cải thiện thuật toán Clustering & TBD \\
      Task 5 & Huấn luyện mô hình phân loại & TBD \\
      Task 6 & Tích hợp Pipeline Streaming & TBD \\
      \bottomrule
    \end{tabular}
    \caption{Tiến độ thực hiện dự án}
  \end{table}
\end{frame}

%------------------------------------------------------------
\section{Tài liệu tham khảo}
\begin{frame}[allowframebreaks]{Tài liệu tham khảo}
  \printbibliography
\end{frame}

\end{document}
