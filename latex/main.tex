\documentclass{beamer}

%========================
% 1) Tiếng Việt (pdfLaTeX)
%========================
\usepackage[utf8]{inputenc}
\usepackage[T5]{fontenc}
\usepackage[vietnamese]{babel}

% Font sans cho slide
\usepackage{helvet}
\renewcommand{\familydefault}{\sfdefault}

%========================
% 2) Theme (giữ Madrid, làm gọn & hiện đại hơn)
%========================
\usetheme{Madrid}
\usecolortheme{dolphin} % tông hiện đại, vẫn "sạch" và an toàn
\setbeamertemplate{navigation symbols}{} % bỏ icon điều hướng cho gọn

% Font/spacing tinh chỉnh
\setbeamerfont{title}{size=\Large,series=\bfseries}
\setbeamerfont{subtitle}{size=\normalsize}
\setbeamerfont{frametitle}{size=\large,series=\bfseries}
\setbeamerfont{block title}{series=\bfseries}
\setlength{\parskip}{4pt}
\setlength{\parindent}{0pt}
\setbeamertemplate{itemize items}[circle]
\setbeamertemplate{enumerate items}[default]
\setbeamercovered{transparent}

%========================
% 3) Packages cần thiết
%========================
\usepackage{booktabs}
\usepackage{graphicx}
\usepackage{ragged2e}
\usepackage{xcolor}
\usepackage{csquotes}
\usepackage{microtype} % chữ "mượt" hơn (pdfLaTeX)

% (caption trong beamer đôi khi gây phiền; nếu không thật sự cần, nên bỏ)
% \usepackage{caption}
\setbeamertemplate{caption}[numbered]

%========================
% 4) LISTINGS: FIX Unicode tiếng Việt cho *mọi* lstlisting
%========================
\usepackage{listings}

% Map tiếng Việt dùng chung (áp cho mọi listings, không chỉ JSON)
\lstdefinestyle{vncode}{
  basicstyle=\tiny\ttfamily,
  breaklines=true,
  showstringspaces=false,
  columns=fullflexible,
  upquote=true,
  literate=
   {á}{{\'a}}1 {à}{{\`a}}1 {ả}{{\h{a}}}1 {ã}{{\~a}}1 {ạ}{{\d{a}}}1
   {ă}{{\u{a}}}1 {ắ}{{\'{\u{a}}}}1 {ằ}{{\`{\u{a}}}}1 {ẳ}{{\h{\u{a}}}}1 {ẵ}{{\~{\u{a}}}}1 {ặ}{{\d{\u{a}}}}1
   {â}{{\^a}}1 {ấ}{{\'{\^a}}}1 {ầ}{{\`{\^a}}}1 {ẩ}{{\h{\^a}}}1 {ẫ}{{\~{\^a}}}1 {ậ}{{\d{\^a}}}1
   {đ}{{\dj}}1 {Đ}{{\DJ}}1
   {é}{{\'e}}1 {è}{{\`e}}1 {ẻ}{{\h{e}}}1 {ẽ}{{\~e}}1 {ẹ}{{\d{e}}}1
   {ê}{{\^e}}1 {ế}{{\'{\^e}}}1 {ề}{{\`{\^e}}}1 {ể}{{\h{\^e}}}1 {ễ}{{\~{\^e}}}1 {ệ}{{\d{\^e}}}1
   {í}{{\'i}}1 {ì}{{\`i}}1 {ỉ}{{\h{i}}}1 {ĩ}{{\~i}}1 {ị}{{\d{i}}}1
   {ó}{{\'o}}1 {ò}{{\`o}}1 {ỏ}{{\h{o}}}1 {õ}{{\~o}}1 {ọ}{{\d{o}}}1
   {ô}{{\^o}}1 {ố}{{\'{\^o}}}1 {ồ}{{\`{\^o}}}1 {ổ}{{\h{\^o}}}1 {ỗ}{{\~{\^o}}}1 {ộ}{{\d{\^o}}}1
   {ơ}{{\.o}}1 {ớ}{{\'{\.o}}}1 {ờ}{{\`{\.o}}}1 {ở}{{\h{\.o}}}1 {ỡ}{{\~{\.o}}}1 {ợ}{{\d{\.o}}}1
   {ú}{{\'u}}1 {ù}{{\`u}}1 {ủ}{{\h{u}}}1 {ũ}{{\~u}}1 {ụ}{{\d{u}}}1
   {ư}{{\.u}}1 {ứ}{{\'{\.u}}}1 {ừ}{{\`{\.u}}}1 {ử}{{\h{\.u}}}1 {ữ}{{\~{\.u}}}1 {ự}{{\d{\.u}}}1
   {ý}{{\'y}}1 {ỳ}{{\`y}}1 {ỷ}{{\h{y}}}1 {ỹ}{{\~y}}1 {ỵ}{{\d{y}}}1
}

% JSON language (chỉ thêm màu sắc, còn Unicode đã được style vncode lo)
\lstdefinelanguage{json}{
  string=[s]{"}{"},
  stringstyle=\color{blue},
  comment=[l]{:},
  commentstyle=\color{black},
}

% Áp style mặc định cho tất cả lstlisting
\lstset{style=vncode}

%========================
% 5) Biblatex (fix cảnh báo vietnamese.lbx)
%========================
\usepackage[backend=biber,style=ieee,language=english]{biblatex}
\addbibresource{references.bib}
\DeclareLanguageMapping{vietnamese}{english}
\setbeamertemplate{bibliography item}[text]

%========================
% 6) Table cosmetics
%========================
\renewcommand{\arraystretch}{1.15}

%------------------------------------------------------------
% THÔNG TIN BÀI THUYẾT TRÌNH (GIỮ ĐÚNG TÊN ĐỀ TÀI)
%------------------------------------------------------------
\title[Phát hiện Xu hướng \& Phân tích Tâm lý]
{Phát hiện Xu hướng và Phân tích Tâm lý theo Thời gian thực}

\subtitle{Báo cáo tiến độ dự án - SE363.Q11}

\author[Tăng Nhất, Lê Minh Nhựt]
{\textbf{Thực hiện:} \break Tăng Nhất\inst{1} \and Lê Minh Nhựt\inst{1} \break \textbf{GVHD}: TS. Đỗ Trọng Hợp\inst{2} \break Nguyễn Ngọc Quí\inst{2}}

\institute[VNU-UIT]{
  \inst{1} Khoa Khoa học Máy tính\\
  Trường Đại học Công nghệ Thông tin

  \inst{2} Khoa Khoa học và Kỹ thuật thông tin \\
  Trường Đại học Công nghệ Thông tin
}

\date[\today]{Ngày 16 Tháng 12 năm 2025}

%------------------------------------------------------------
\begin{document}

\frame{\titlepage}

\AtBeginSection[]
{
  \begin{frame}
    \frametitle{Mục lục}
    \tableofcontents[currentsection]
  \end{frame}
}

\begin{frame}
  \frametitle{Nội dung báo cáo}
  \tableofcontents
\end{frame}

%------------------------------------------------------------
\section{Tổng quan dự án}
\begin{frame}{Mục tiêu đề tài}
  \justifying
  Xây dựng hệ thống \textbf{phát hiện xu hướng/sự kiện theo thời gian thực} từ dữ liệu mạng xã hội và báo chí,
  nhằm rút ngắn khoảng cách giữa \textit{dữ liệu thô} và \textit{thông tin có thể hành động}.

  \begin{block}{Hai nhóm đối tượng chính}
    \begin{itemize}
      \item \textbf{Chính phủ/An toàn công cộng:} Phát hiện sớm rủi ro xã hội, thiệt hại thiên tai, biểu tình, tin giả, ... (\textit{Social Risk}).
      \item \textbf{Doanh nghiệp/Marketing:} Nắm bắt nhanh xu hướng tiêu dùng, viral trends, ... (\textit{Market Opportunity}).
    \end{itemize}
  \end{block}

  \textbf{Hướng tiếp cận:} Kết hợp \textbf{tín hiệu đa nguồn} (Search--Social--News) và NLP để \textbf{gom topic}, \textbf{chấm điểm xu hướng} từ đó nắm bắt được tình trạng, phản ứng của mọi người về các xu hướng mới nhất.
\end{frame}

%============================================================
\begin{frame}{Ví dụ cụ thể: Theo dõi phản ứng xã hội theo thời gian thực}
  \justifying
  \begin{block}{Ví dụ: Dự kiến đánh thuế hộ kinh doanh thu nhập 200 triệu/năm}
    Thông tin dự kiến đánh thuế được lan truyền nhanh trên mạng xã hội,
    kèm theo nhiều cách diễn đạt khác nhau và cảm xúc trái chiều.
  \end{block}

  \begin{itemize}
    \item Mục tiêu hệ thống:
          \begin{itemize}
            \item \textbf{Tóm tắt sự việc:} từ thông tin ban đầu $\rightarrow$ lan truyền $\rightarrow$ phản hồi chính thức.
            \item \textbf{Cảm xúc chủ đạo của người dân:} lo lắng, bức xúc, hoang mang.
            \item \textbf{Mức độ lan rộng:} thảo luận tăng mạnh trong thời gian ngắn.
          \end{itemize}

          % \item Khi kết hợp với báo chí:
          %       \begin{itemize}
          %         \item Xác định thời điểm thông tin được \textbf{đính chính / làm rõ}.
          %       \end{itemize}
  \end{itemize}

  \begin{alertblock}{Ý nghĩa}
    Giúp cơ quan quản lý \textbf{nắm bắt phản ứng xã hội sớm},
    chủ động truyền thông để tránh hiểu sai và giảm bức xúc dư luận.
  \end{alertblock}
\end{frame}


\begin{frame}{Phân loại sự kiện (Taxonomy)}
  \justifying
  Hệ thống tập trung vào 7 loại sự kiện/xu hướng chính (phục vụ lọc nhiễu và gán nhãn đánh giá):
  \begin{enumerate}
    \item \textbf{Social Controversy:} Tranh cãi xã hội, bê bối (Ưu tiên cao cho CP).
    \item \textbf{Civil Unrest:} Biểu tình, đình công.
    \item \textbf{Natural Disaster:} Thiên tai, lũ lụt, môi trường.
    \item \textbf{Public Safety:} Tai nạn, cháy nổ.
    \item \textbf{Politics \& Policy:} Bầu cử, phản ứng chính sách.
    \item \textbf{Viral Lifestyle:} Xu hướng ăn uống, thời trang (Ưu tiên cho Marketing).
    \item \textbf{Entertainment:} Giải trí, văn hóa đại chúng.
  \end{enumerate}

  \vspace{0.25em}
  \begin{alertblock}{Vì sao cần đa nguồn?}
    Search phản ánh \textit{ý định}, Social phản ánh \textit{độ lan truyền} nhưng nhiễu, News phản ánh \textit{tính xác thực} nhưng có độ trễ.
  \end{alertblock}
\end{frame}

%------------------------------------------------------------
\section{Dữ liệu \& Thu thập}

%============================================================
\begin{frame}[fragile]{Nguồn dữ liệu 1: Mạng xã hội (Facebook)}
  \justifying
  \begin{itemize}
    \item \textbf{Nguồn:} Fanpage lớn / trang tin Facebook (Theanh28, Dân trí, \ldots).
    \item \textbf{Định dạng:} JSON (nội dung + thời gian đăng + tương tác).
    \item \textbf{Tín hiệu trend:} likes / comments / shares.
    \item \textbf{Trạng thái:} Đã crawl và đưa về schema thống nhất.
  \end{itemize}

  \begin{alertblock}{Hạn chế}
    Crawl thủ công \textbf{không luôn lấy được giờ/phút chính xác} với bài đăng cũ
    $\rightarrow$ ảnh hưởng đánh giá xu hướng theo thời gian.
  \end{alertblock}

  \begin{exampleblock}{Schema hiện tại (rút gọn)}
    {\footnotesize
      \begin{verbatim}
{
  "page_name": "Theanh28",
  "published_time": "2025-12-15T01:54:10",
  "content": "...",
  "stats": { "likes": 3800, "comments": 99, "shares": 10 }
}
\end{verbatim}
    }
  \end{exampleblock}
\end{frame}


%============================================================
\begin{frame}[fragile]{Nguồn dữ liệu 1: Facebook – Định hướng nâng chất lượng (Apify)}
  \justifying
  \begin{block}{Mục tiêu}
    Sử dụng \textbf{Apify} để crawl Facebook với \textbf{timestamp chính xác}
    (giờ/phút/giây) cho cả bài cũ.
  \end{block}

  \begin{itemize}
    \item Lấy được \textbf{time ISO + epoch timestamp}.
    \item Giữ thông tin postId, url, pageName, link bài báo.
    \item Đầy đủ tương tác: likes / comments / shares.
  \end{itemize}

  \begin{exampleblock}{Schema Apify (các trường chính)}
    {\footnotesize
      \begin{verbatim}
{
  "pageName": "baodantridientu",
  "postId": "1191254559782876",
  "time": "2025-12-15T09:41:09.000Z",
  "timestamp": 1765791669,
  "text": "...",
  "likes": 9, "comments": 1, "shares": 1
}
\end{verbatim}
    }
  \end{exampleblock}

  \begin{exampleblock}{Lợi ích}
    Tính \textbf{engagement theo thời gian} chính xác hơn
    $\rightarrow$ phát hiện trend sớm hơn.
  \end{exampleblock}
\end{frame}



\begin{frame}[fragile]{Nguồn dữ liệu 2: Tin tức (News)}
  \justifying
  \begin{itemize}
    \item \textbf{Nguồn:} Báo Thanh Niên, Tuổi Trẻ, VnExpress, Vietnamnet, ...
    \item \textbf{Định dạng:} CSV (title, content, publish time, url, ...).
    \item \textbf{Vai trò:} Kiểm chứng sự kiện + tín hiệu \textbf{mật độ đưa tin} theo thời gian.
    \item \textbf{Trạng thái:} \textbf{Đã crawl} và hợp nhất vào pipeline chung.
  \end{itemize}


  \begin{exampleblock}{Mẫu dữ liệu bài báo (News)}
    {\scriptsize
      \begin{table}
        \centering
        \begin{tabular}{p{1.1cm} p{1.8cm} p{1.8cm} p{2.5cm} p{1.1cm}}
          \toprule
          \textbf{ID}                                                                                                             & \textbf{URL} & \textbf{Tiêu đề} & \textbf{Nội dung} & \textbf{Thời gian} \\
          \midrule
          7287b...                                                                                                                &
          vietnamnet.vn/...                                                                                                       &
          Nhật Bản đau đầu với ngân sách                                                                                          &
          Ngân khố của chính phủ Nhật Bản đã nhận được 129 tỷ Yen từ các nguồn thu bất thường, gây tranh luận lớn trong nội bộ... &
          09/12/2025 10:37                                                                                                                                                                                   \\
          \bottomrule
        \end{tabular}
      \end{table}
    }
  \end{exampleblock}

\end{frame}

%------------------------------------------------------------
\section{Phương pháp \& Kiến trúc dự kiến}
\begin{frame}{Kiến trúc hệ thống dự kiến}
  \justifying
  Hệ thống chia làm 2 pha chính (phục vụ phát hiện xu hướng và vận hành thời gian thực):
  \begin{columns}
    \column{0.5\textwidth}
    \textbf{Phase 1: Offline Discovery}
    \begin{itemize}
      \item Thu thập dữ liệu lịch sử đa nguồn.
      \item Chuẩn hoá \& tiền xử lý tiếng Việt.
      \item Vector hoá (embedding) và \textbf{clustering}.
      \item Tạo luật/gán nhãn bán giám sát để phân loại trend.
    \end{itemize}

    \column{0.5\textwidth}
    \textbf{Phase 2: Online Real-time}
    \begin{itemize}
      \item Thu thập định kỳ/streaming.
      \item Ghép bài viết vào topic + phát hiện topic mới.
      \item Chấm điểm xu hướng theo thời gian.
      \item Cảnh báo \& lưu trữ cho dashboard.
    \end{itemize}
  \end{columns}

  \vspace{0.35em}
  \begin{block}{Trọng tâm hiện tại}
    \textbf{Phát hiện xu hướng} (semantic matching + clustering + lọc nhiễu) là phần khó và tốn công nhất.
  \end{block}
\end{frame}

\begin{frame}{Pipeline NLP phát hiện xu hướng (đang phát triển)}
  \justifying
  \begin{block}{Mục tiêu kỹ thuật}
    Gom các biểu đạt khác nhau (Facebook/News/Search) nhưng cùng một sự kiện thành \textbf{một topic thống nhất}.
  \end{block}

  \begin{itemize}
    \item \textbf{Tiền xử lý:} chuẩn hoá Unicode, loại boilerplate, làm sạch ký tự/emoji.
    \item \textbf{Biểu diễn ngữ nghĩa:} dùng \textbf{embedding tiếng Việt} (PhoBERT / Vietnamese bi-encoder).
    \item \textbf{Tăng cường ngữ cảnh (2 hướng thử nghiệm):}
          \begin{itemize}
            \item \textbf{NER (underthesea):} trích xuất thực thể để tăng độ khớp khi gom nhóm.
            \item \textbf{Alias từ Google Trends:} mở rộng biến thể từ khoá để tăng recall khi khác cách diễn đạt.
          \end{itemize}
    \item \textbf{Ghép \& gom topic:} cosine similarity (matching) + UMAP (giảm chiều) + HDBSCAN (clustering).
  \end{itemize}
\end{frame}

\begin{frame}{So sánh trực quan: Clustering với NER}
  \justifying

  \begin{minipage}{\linewidth}
    \begin{block}{Thiết lập}
      {\footnotesize
        Embedding tiếng Việt (PhoBERT / bi-encoder) + \textbf{NER enrichment (underthesea)}
        $\rightarrow$ UMAP + HDBSCAN.
      }
    \end{block}
  \end{minipage}

  \begin{figure}
    \centering
    \includegraphics[width=0.55\textwidth]{assets/images/ner.png}
    \caption{\footnotesize Phân bố cụm khi tăng cường ngữ cảnh bằng Named Entity Recognition}
  \end{figure}

  % {\footnotesize
  % \begin{itemize}
  %   \item Các bài viết cùng \textbf{thực thể chính} có xu hướng gom lại tốt hơn.
  %   \item Tuy nhiên vẫn xuất hiện \textbf{cụm lớn do nội dung lặp/ngày}.
  % \end{itemize}
  % }
\end{frame}


\begin{frame}{So sánh trực quan: Clustering với Alias từ Google Trends}
  \justifying

  \begin{minipage}{\linewidth}
    \begin{block}{Thiết lập}
      {\footnotesize
        Embedding tiếng Việt (PhoBERT / bi-encoder) + \textbf{Alias normalization từ Google Trends}
        $\rightarrow$ UMAP + HDBSCAN.
      }
    \end{block}
  \end{minipage}

  \begin{figure}
    \centering
    \includegraphics[width=0.55\textwidth]{assets/images/ggtrend.png}
    \caption{\footnotesize Phân bố cụm khi mở rộng biến thể từ khoá từ Google Trends}
  \end{figure}

  % {\footnotesize
  % \begin{itemize}
  %   \item Tăng \textbf{độ bao phủ (recall)} cho các sự kiện có nhiều cách gọi.
  %   \item Một số cụm rõ ràng hơn nhưng vẫn cần \textbf{lọc nhiễu bổ sung}.
  % \end{itemize}
  % }
\end{frame}



\begin{frame}{Trend Scoring \& Phân loại trend (theo đề xuất)}
  \justifying
  Với mỗi \textbf{topic}, hệ thống tổng hợp 3 tín hiệu:
  \begin{itemize}
    \item \textbf{Google Search Score (G):} mức quan tâm/tăng trưởng tìm kiếm.
    \item \textbf{Facebook Engagement Score (F):} mức tương tác được chuẩn hoá theo thời gian.
    \item \textbf{News Coverage Score (N):} mật độ bài báo theo thời gian.
  \end{itemize}

  \begin{block}{Unified Trend Score}
    \[
      T = w_G \cdot G + w_F \cdot F + w_N \cdot N
    \]
    Trọng số có thể điều chỉnh theo mục tiêu (ưu tiên viral / xác thực / ý định tìm kiếm).
  \end{block}

  \begin{exampleblock}{Phân loại xu hướng}
    Search-only / Social-only / News-only / Multi-source confirmed / Emerging / Fading.
  \end{exampleblock}
\end{frame}

\begin{frame}{Thay đổi \& Cập nhật mô hình}
  \justifying
  Sau quá trình thử nghiệm (08/12/2025), nhóm đã thực hiện các điều chỉnh:

  \begin{alertblock}{Vấn đề gặp phải}
    Dữ liệu \textbf{lặp hàng ngày} (thời tiết, giá vàng, xổ số, \ldots) xuất hiện dày đặc, tạo \textbf{topic giả} và gây nhiễu lớn.
  \end{alertblock}

  \begin{block}{Giải pháp điều chỉnh (đã/đang áp dụng)}
    \begin{itemize}
      \item \textbf{Embedding tiếng Việt:} ưu tiên PhoBERT/Vietnamese bi-encoder để ghép ngữ nghĩa ổn định.
      \item \textbf{Bổ sung NER:} tăng độ chính xác khi gom nhóm + hỗ trợ lọc rác theo thực thể.
      \item \textbf{Bổ sung alias:} khai thác bộ từ khoá liên quan từ Google Trends để tăng khả năng match.
    \end{itemize}
  \end{block}
\end{frame}

% === SLIDE MỚI THEO YÊU CẦU ===
\begin{frame}{Kết quả thử nghiệm bước đầu}
  \justifying
  \begin{columns}
    \column{0.52\textwidth}
    \begin{block}{Những điểm đã đạt được}
      \begin{itemize}
        \item Đã chạy được pipeline: \textbf{chuẩn hoá} $\rightarrow$ \textbf{embedding} $\rightarrow$ \textbf{matching} $\rightarrow$ \textbf{clustering}.
        \item Embedding tiếng Việt cho thấy khả năng gom các câu gần nghĩa tốt hơn so với so khớp từ khoá thuần.
        \item Alias từ Google Trends giúp tăng độ bao phủ khi cùng sự kiện nhưng khác cách gọi.
      \end{itemize}
    \end{block}

    \column{0.48\textwidth}
    \begin{alertblock}{Vấn đề còn tồn tại}
      \begin{itemize}
        \item \textbf{Cụm lớn bất thường} do nhiễu (daily/spam) ``nuốt'' dữ liệu.
        \item \textbf{Overlap giữa cụm} khi nội dung na ná (bán hàng, câu view).
        \item Ngưỡng similarity nhạy: thay đổi nhỏ có thể làm lệch tỉ lệ matched/unmatched.
      \end{itemize}
    \end{alertblock}
  \end{columns}

  \vspace{0.35em}
  \begin{exampleblock}{Hướng cải thiện ngay}
    Lọc nhiễu theo chu kỳ + từ khoá đặc trưng, tuning UMAP/HDBSCAN, và bổ sung đánh giá chất lượng cụm (kết hợp thủ công + chỉ số nội bộ).
  \end{exampleblock}
\end{frame}

%------------------------------------------------------------
\section{Khó khăn \& Kế hoạch}
\begin{frame}{Khó khăn hiện tại}
  \justifying

  \textit{Trong thực tế, việc phát hiện một xu hướng ``thật sự'' là bài toán khó,
    do các thảo luận trên mạng xã hội thường bùng phát nhanh, nhiễu nhiều,
    và không phải lúc nào mức độ lan truyền cũng phản ánh đúng tầm quan trọng của sự kiện.}

  \vspace{0.5em}

  \begin{enumerate}
    \item \textbf{Vấn đề Phân cụm (Clustering):}
          \begin{itemize}
            \item Chất lượng phân cụm chưa cao; topic bị chồng lấn.
            \item Một số cụm quá lớn do ``nuốt'' dữ liệu nhiễu (daily/spam).
          \end{itemize}
    \item \textbf{Xử lý nhiễu (Noise filtering):}
          \begin{itemize}
            \item Chưa có cơ chế triệt để loại bỏ tin lặp (giá vàng, thời tiết, xổ số, \ldots).
            \item Cần kết hợp rule-based (chu kỳ/từ khoá) và thống kê theo thời gian.
          \end{itemize}
    \item \textbf{Đồng bộ thời gian đa nguồn:}
          \begin{itemize}
            \item Social nhanh, News có độ trễ, Search phản ánh ý định theo ngữ cảnh.
            \item Cần chuẩn hoá theo \textbf{time-binning} để chấm điểm công bằng.
          \end{itemize}
  \end{enumerate}
\end{frame}


\begin{frame}{Kế hoạch tiếp theo (Timeline)}
  \justifying
  \begin{table}
    \centering
    \small
    \begin{tabular}{p{0.12\textwidth} p{0.62\textwidth} p{0.20\textwidth}}
      \toprule
      \textbf{Task} & \textbf{Mô tả}                                               & \textbf{Trạng thái}           \\
      \midrule
      Task 1        & Chuẩn hoá schema + làm sạch cơ bản (Unicode/boilerplate)     & Đang làm                      \\
      Task 2        & Thu thập dữ liệu (Crawler)                                   & Đã Hoàn thành các bộ Crawl    \\
      Task 3        & Semantic matching (embedding) + alias/NER enrichment         & Đang thử nghiệm               \\
      Task 4        & Tối ưu Clustering (UMAP + HDBSCAN) + đánh giá chất lượng cụm & Đang tìm giải pháp tối ưu hơn \\
      Task 5        & Trend scoring + phân loại trend đa nguồn                     & TBD                           \\
      Task 6        & Dashboard/báo cáo + tích hợp pipeline định kỳ/streaming      & TBD                           \\
      \bottomrule
    \end{tabular}
    \caption{Tiến độ thực hiện dự án}
  \end{table}
\end{frame}

%------------------------------------------------------------
\section{Tài liệu tham khảo}
\begin{frame}[allowframebreaks]{Tài liệu tham khảo}
  \printbibliography
\end{frame}

\end{document}
