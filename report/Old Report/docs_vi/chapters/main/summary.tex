\chapter*{\centering\Large{Tóm tắt đồ án}}
\addcontentsline{toc}{chapter}{Tóm tắt đồ án}

Trong thời đại quá tải thông tin, việc phát hiện và hiểu nhanh các sự kiện mới nổi từ luồng mạng xã hội là vô cùng quan trọng cho an toàn công cộng và thông tin thị trường.
Tuy nhiên, các hệ thống thực tế phải đối mặt với hai thách thức cốt lõi: (i) \textit{sự mơ hồ ngữ nghĩa} trong các bài đăng ngắn, nhiễu, và (ii) \textit{khoảng trống đánh giá} do thiếu nhãn trong dữ liệu luồng.
Chúng tôi trình bày một \textbf{hệ thống phát hiện sự kiện thời gian thực} được xây dựng trên chiến lược \textbf{Phân cụm Phân cấp Nhận thức Mạng xã hội (SAHC)} mới lạ và \textbf{kiến trúc hai đường} kết hợp Spark/Kafka cho phát hiện độ trễ thấp với các worker LLM bất đồng bộ cho tinh chỉnh ngữ nghĩa sâu.
Trên tập con \textbf{Mini-Ground Truth} (N=300 bài đăng được gán nhãn thủ công), pipeline lai đầy đủ đạt \textbf{NMI = 0.5978}.
Trên bộ dữ liệu đầy đủ gồm 7,605 mục, chúng tôi báo cáo các chỉ số phân cụm nội tại (Silhouette, Davies--Bouldin, tỷ lệ nhiễu) và giao thức \textbf{LLM-as-a-Judge} để đánh giá tính nhất quán ngữ nghĩa ở quy mô lớn.

\textbf{Từ khóa:} Phát hiện Sự kiện, Khai phá Mạng Xã hội, Xử lý Luồng Thời gian Thực, Mô hình Ngôn ngữ Lớn, Phân cụm Lai.
